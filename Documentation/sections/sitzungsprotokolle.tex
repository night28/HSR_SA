\section{Sitzungsprotokolle}
%\includepdf[scale=1.0,pages={1}]{pdfincludes/sitzungsprotokolle}

\subsection{Sitzungsprotokoll 27.02.2018}

\subsubsection{Sitzungsteilnehmer}
\begin{itemize}	
	\item Laurent Metzger 
	\item Philipp Albrecht
	\item Sandro Kaspar
	\item Jessica Kalberer
\end{itemize}

\subsubsection{Traktanden}
\begin{itemize}	
	\item Projektstart
	\item Besprechung genaue Aufgabenstellung und nächste Schritte
\end{itemize}

\subsection{Beschlüsse (Diskussion)}
\begin{itemize}	
	\item Evaluieren eines Software Defined Network im Campus Bereich für FUB.
	\item Anleitung für FUB für die Erstellung eines SD Networks mittels DNA Center.
	\item Freie Hand bei Gestaltung wöchentlicher Reports, da nicht alle Möglichkeiten bekannt.
	\item Geräte werden erst Mitte März 2018 geliefert
	\item Offene Frage: Vorgaben auf welcher Plattform Projekt laufen soll (Dropbox, gitHub)?
\end{itemize}

\subsubsection{Offene Punkte (erledigt vor nächster Sitzung)}
\begin{table}[H]
	\rowcolors{2}{gray!25}{white}
	\centering
	\begin{tabularx}{\textwidth}{X | p{4.5cm}}
		\rowcolor{gray!50}
		\textbf{Was} & \textbf{Verantwortlichkeit} \\
		\hline	
		Projektplan mit Meilensteinen erstellen	& Philipp \\	
		Sitzungsprotokoll vom 27.02.2018 erstellen & Jessica \\
		Beschreibung der SD-A Lösung mit Vorteilen im Vergleich zu klassischem Campus Design (Management Summary) &	Sandro \\
		Module 2 Lesson 2 auf Cisco Learning Library anschauen (Part 1 und Part 2) & Philipp, Sandro, Jessica \\
		Dokumentation vorbereiten (Latex) anhand Strukturierungsbeispiel 2 & Jessica \\
		Zeiterfassung Tool vorbereiten & Jessica \\	
	\end{tabularx}
	\label{tab:my-label}
\end{table}

\subsubsection{Nächster Termin}
\begin{itemize}	
	\item Meeting mit Betreuer: 06. März 2018, 10 Uhr, 60 Minuten
	\item Meeting mit Industriepartner: 08. März 2017, 14 Uhr, 120 Minuten
\end{itemize}

\subsubsection{Kommende Abwesenheiten}
keine



\subsection{Sitzungsprotokoll 06.03.2018}

\subsubsection{Sitzungsteilnehmer}
\begin{itemize}	
	\item Laurent Metzger 
	\item Urs Baumann 
	\item Philipp Albrecht
	\item Sandro Kaspar
	\item Jessica Kalberer
\end{itemize}

\subsubsection{Traktanden}
\begin{itemize}	
	\item Aufgabenstellung schriftlich vom Betreuer erhalten? Bekommen wir diese noch?
	\begin{itemize}
		\item erhalten wir in den letzten zwei Wochen
	\end{itemize}
	\item Zeiterfassung mit Toggl / Waffle.io / GitHub Issues so sinnvoll oder anders gewünscht?
	\begin{itemize}
		\item Tools passen, jedoch den Betreuern noch Zugang zu allen Tools geben
	\end{itemize} 
	\item Business Dresscode für Besprechung mit Industriepartner gewünscht?
	\begin{itemize}
		\item Nein, normale anständige Kleidung reicht
	\end{itemize}
	\item Teilnehmer Besprechung Industriepartner und deren Rollen?
	\begin{itemize}
		\item FUB Leiter vom Netzwerk mit einem Mitarbeiter
	\end{itemize}
	\item Was muss für die Besprechung mit dem Industriepartner vorbereitet werden?
	\begin{itemize}
		\item wir werden in erster Linie Informationen von FUB erhalten
		\item Grafik vorbereiten um eine Übersicht über unsere Tools zu zeigen
	\end{itemize}
	\item Arbeit auf GitHub private oder public? Waffle.io wenn private 5 Dollar / Monat
	\begin{itemize}
		\item Industriepartner am Donnerstag nochmals darauf ansprechen
	\end{itemize}
	\item Technologien einzeln genauer beschreiben notwendig?
	\begin{itemize}
		\item Technologien im technischen Bericht genauer beschreiben (SDA, DNA,..)
	\end{itemize}
\end{itemize}

\subsection{Beschlüsse (Diskussion)}
\begin{itemize}	
	\item Use Cases Bereiche (ca. 10 Use Cases generieren). Unterscheidung welche Änderung das DNA Center bringt. Welche Use Cases sind neu? Use Cases müssen anfangs nicht komplett ins Detail beschrieben werden. Vielleicht zuerst User Stories generieren und daraus dann die Use Cases ableiten. Diese können dann mit Industriepartner abgeglichen werden, ob diese mit ihm übereinstimmen. Beispiele für zwei Use Cases:
	\begin{itemize}
		\item Definierung von Benutzer- und Geräteprofile, um basierend auf Geschäftsanforderungen die Zugriffsrechte und Netzwerksegmentierung zu verwalten und so das Netzwerk sicher zu halten
		\item Durch APIs, Erstellung von wöchentlichen Reports per E-Mail
	\end{itemize}
	\item GitHub private oder public?
	\begin{itemize}
		\item Wird mit Industriepartner am nächsten Donnerstag direkt abgeklärt, aber wahrscheinlich ist es egal das wir es public machen
		\item Zugriffe für GitHub, Toggl, Waffle.io für Betreuer einrichten
	\end{itemize} 
	\item Technologien welche für unsere Arbeit essentiell sind im technischen Bericht festhalten, wie beispielsweise DNA Center, VXLAN, LISP. Doch Technologien wie BGP müssen nicht weiter dokumentiert werden, da genügend Cisco Quellen verfügbar sind und bekannt sein sollte.
	\item Projektmanagement gewünschter Inhalt:
	\begin{itemize}
		\item Projektplan
		\item Arbeitspakete
		\item Risikomanagement
		\item Testprotokoll (um Use Cases zu überprüfen)
	\end{itemize}
	\item Sitzung am Donnerstag mit Industriepartner für uns erst um 15:30 Uhr
	\begin{itemize}
		\item Dresscode für Meeting normal wie immer
		\item Präsentation mit Industriepartner Dresscode edel erwünscht mit Hemd etc.
	\end{itemize}
	\item Netzwerk-Umgebung: es muss noch eine passende Netzwerk-Topologie erstellt werden
	\begin{itemize}
		\item Hardware
		\begin{itemize}
			\item 4 x Catalyst 9300
			\item 4 x Catalyst 3850
		\end{itemize}
		\item VMs werden von Betreuer erstellt und wir erhalten VPN Zugriff auf die Server, falls wir Hardware Zugriff benötigen, befinden sich die Switches im Netzwerklabor.
		\begin{itemize}
			\item ISE (Betreuer)
			\item Infobox
			\item DHCP (Ubuntu VM)
			\item DNS (Ubuntu VM)
			\item NTP (Ubuntu VM)
		\end{itemize}
	\end{itemize}
	\item Traktanden jeweils am Montagabend vorher an Betreuer senden.
	\item Kosten des Projektes
	\begin{itemize}
		\item Hardware DNA Center um die 90'000 Fr
		\item Switch je à 10'000 Fr
		\item grundsätzlich wird alles von Urs im Netzwerklabor installiert
		\item Softwaretechnisch kann alles an Cisco retourniert werden, wenn etwas nicht mehr bootet
	\end{itemize}
\end{itemize}

\subsubsection{Offene Punkte (erledigt vor nächster Sitzung)}
\begin{table}[H]
	\rowcolors{2}{gray!25}{white}
	\centering
	\begin{tabularx}{\textwidth}{X | p{4.5cm}}
		\rowcolor{gray!50}
		\textbf{Was} & \textbf{Verantwortlichkeit} \\
		\hline	

		Zugriffe auf GitHub, Waffle.io und Toggl an Betreuer senden & Sandro \\
		Grafik vorbereiten für Übersicht über unsere Tools & Philipp \\
		GitHub private oder public mit FUB abklären am Donnerstag & Philipp, Sandro, Jessica \\
		Eingesetzte Technologien dokumentieren & Jessica \\
		Netzwerk-Topologie Vorschlag & Philipp \\
		Risiko-Management Tabelle & Sandro \\
		Use Cases vorbereiten (ca. 10 Use Cases generieren) & Philipp, Sandro, Jessica \\
		Sitzungsprotokoll in Latex übernehmen & Jessica \\
		Sitzungsprotokoll Traktanden jeweils spätestens Montagabend an Betreuer	& Jessica \\
		Testprotokoll Vorlage erstellen anhand von Use Cases & Jessica \\

	\end{tabularx}
	\label{tab:my-label}
\end{table}

\subsubsection{Nächster Termin}
\begin{itemize}	
	\item Sitzung mit Industriepartner: 08. März 2018, 15.30 Uhr, 30 Minuten
	\item Sitzung mit Betreuer: 13. März 2018, 15.10 Uhr, 60 Minuten
\end{itemize}

\subsubsection{Kommende Abwesenheiten}
keine

\newpage

\subsection{Sitzungsprotokoll 08.03.2018}

\subsubsection{Sitzungsteilnehmer}
\begin{itemize}	
	\item Laurent Metzger 
	\item Urs Baumann
	\item Philipp Albrecht
	\item Sandro Kaspar
	\item Jessica Kalberer
\end{itemize}

\subsubsection{Traktanden}
\begin{itemize}	
	\item Arbeit auf GitHub private oder public? Waffle.io wenn private 5 Dollar / Monat
	\begin{itemize}
		\item Industriepartner am Donnerstag nochmals darauf ansprechen
	\end{itemize}
\end{itemize}

\subsection{Beschlüsse (Diskussion)}
\begin{itemize}	
	\item x
	\item x
\end{itemize}

\subsubsection{Offene Punkte (erledigt vor nächster Sitzung)}
\begin{table}[H]
	\rowcolors{2}{gray!25}{white}
	\centering
	\begin{tabularx}{\textwidth}{X | p{4.5cm}}
		\rowcolor{gray!50}
		\textbf{Was} & \textbf{Verantwortlichkeit} \\
		\hline	
		x & x \\	
	\end{tabularx}
	\label{tab:my-label}
\end{table}

\subsubsection{Nächster Termin}
\begin{itemize}	
	\item Meeting mit Betreuer: 13. März 2018, 15.10 Uhr, 60 Minuten
\end{itemize}

\subsubsection{Kommende Abwesenheiten}
keine

\newpage

\subsection{Sitzungsprotokoll 13.03.2018}

\subsubsection{Sitzungsteilnehmer}
\begin{itemize}	
	\item Laurent Metzger 
	\item Urs Baumann
	\item Philipp Albrecht
	\item Sandro Kaspar
	\item Jessica Kalberer
\end{itemize}

\subsubsection{Traktanden}
\begin{itemize}	
	\item x
	\item x
\end{itemize}

\subsection{Beschlüsse (Diskussion)}
\begin{itemize}	
	\item x
	\item x
\end{itemize}

\subsubsection{Offene Punkte (erledigt vor nächster Sitzung)}
\begin{table}[H]
	\rowcolors{2}{gray!25}{white}
	\centering
	\begin{tabularx}{\textwidth}{X | p{4.5cm}}
		\rowcolor{gray!50}
		\textbf{Was} & \textbf{Verantwortlichkeit} \\
		\hline	
		x & x \\	
	\end{tabularx}
	\label{tab:my-label}
\end{table}

\subsubsection{Nächster Termin}
\begin{itemize}	
	\item Meeting mit Betreuer: 20. März 2018, 15.10 Uhr, 60 Minuten
\end{itemize}

\subsubsection{Kommende Abwesenheiten}
keine



