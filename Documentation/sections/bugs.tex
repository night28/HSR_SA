\section{Bugs}
Im folgenden werden Bugs aufgeführt, die während unserer Arbeit aufgetreten sind. 

\newcommand{\bugreport}[9]{
\subsubsection{#1}
	\begin{table}[H]
		\rowcolors{2}{gray!25}{white}
		\centering
		\begin{tabularx}{\textwidth}{l | X}
			Komponente   & #2       \\
			Dringlichkeit   & #3       \\
			\hline
			Beschreibung   & #4  \\ 
			\hline
			Konsequenzen   & #5  \\ 
			\hline
			Workaround & #6 \\
			\hline
			Reproduzieren & 
			#7
			\\
			\hline
			Reporter  & #8 \\
			Feedback Cisco & #9 \\
		\end{tabularx}
		\caption{Bug: #1}
	\end{table}
}

\bugreport
% Titel
{Backup Server hinzufügen}
% Komponente
{Einstellungen $\rightarrow$ System Settings $\rightarrow$ Backup \& Restore}
% Dringlichkeit
{Hoch}
% Beschreibung
{Nach der Eingabe der Backup Server Einstellungen und den klick auf \textit{Apply} geschieht nichts. Nach einer Weile werden immer mehr Docker Container gestoppt, bis das DNA Center nicht mehr gebraucht werden kann. Ein Neustart ist erforderlich}
% Konsequenzen
{Beim Ausfall der Appliance können die Einstellungen nicht wiederhergestellt werden}
% Workaround
{Keiner}
%Reproduzieren
{
	\begin{enumerate}
		\item Einstellungen $\rightarrow$ System Settings $\rightarrow$ Backup \& Restore
		\item Im Popup \textit{Configure} wählen. 
		\item SSH Servereinstellungen eingeben
		\item \textit{Apply} drücken. 
	\end{enumerate}
}
% Reporter
{Sandro Kaspar}
% Feedback Cisco
{}

\bugreport
% Titel
{Netzwerkgerät OS Update}
% Komponente
{Provision $\rightarrow$ Devices $\rightarrow$ Inventory}
% Dringlichkeit
{Mittel}
% Beschreibung
{Im DNA Center können OS Images automatisch auf Netzwerkgeräte aufgespielt werden. Dieser Funktion hat bei allen Versuchen immer zu Fehler geführt. 
\textbf{Wichtig:}
Im Imagerepository muss das Image verfügbar sein. }
% Konsequenzen
{OS Updates müssen manuell durchgeführt werden.}
% Workaround
{Update manuell via TFTP direkt via CLI auf dem Netzwergerät ausführen.}
%Reproduzieren
{
	\begin{enumerate}
		\item Provision $\rightarrow$ Devices $\rightarrow$ Inventory
		\item Gewünschtes Gerät anwählen 
		\item Action $\rightarrow$ Update OS Image
		\item Im Popup Gerät auswählen $\rightarrow$ Update
	\end{enumerate}
}
% Reporter
{Sandro Kaspar}
% Feedback Cisco
{}

\bugreport
% Titel
{DNA Center Update - Appliance nicht nutzbar während Update}
% Komponente
{Einstellungen $\rightarrow$ App Management}
% Dringlichkeit
{Mittel}
% Beschreibung
{Wenn ein \textit{System Update} oder ein \textit{Package Update} durchgeführt wird, kann das GUI des DNA Center nicht verwendet werden. Problematisch: Der Zugriff ist trotzdem möglich. Jedoch sind dann zufällig Funktionen nicht vorhanden oder benutzbar.}
% Konsequenzen
{Appliance nicht nutzbar während Update}
% Workaround
{Während Update GUI nicht verwenden.}
%Reproduzieren
{
	Bedingung: Updates sind Verfügbar.
	\begin{enumerate}
		\item Einstellungen $\rightarrow$ App Management
		\item \textit{System Update} oder \textit{Package Update} wählen.
		\item Packages auswählen
		\item Install oder Update wählen.
	\end{enumerate}
}
% Reporter
{Philipp Albrecht}
% Feedback Cisco
{Siehe Workaround}







