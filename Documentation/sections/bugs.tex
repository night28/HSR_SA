\section{Bugs}
Im folgenden werden Bugs aufgeführt, die während unserer Arbeit aufgetreten sind. Die hier erwähnten Bugs wurden auch via Slack an Cisco gemeldet. Alle erwähnten Bugs beziehen sich auf das DNA Center in Version 1.1.6 und ISE in Version 2.3.

\newcommand{\bugreport}[9]{
\subsection{#1}
	\begin{table}[H]
		\rowcolors{2}{gray!25}{white}
		\centering
		\begin{tabularx}{\textwidth}{| l | X |}
			\hline
			Komponente   & #2       \\
			\hline
			Dringlichkeit   & #3       \\
			\hline
			Beschreibung   & #4  \\ 
			\hline
			Konsequenzen   & #5  \\ 
			\hline
			Workaround & #6 \\
			\hline
			Reproduzieren & #7	\\
			\hline
			Reporter  & #8 \\
			\hline
			Feedback Cisco & #9 \\
			\hline
		\end{tabularx}
		\caption{Bug: #1}
	\end{table}
}

\bugreport
% Titel
{Backup Server hinzufügen}
% Komponente
{Einstellungen $\rightarrow$ System Settings $\rightarrow$ Backup \& Restore}
% Dringlichkeit
{Hoch}
% Beschreibung
{Nach der Eingabe der Backup Server Einstellungen und dem Klick auf \textit{Apply} geschieht nichts. Nach einer Weile werden immer mehr Docker Container gestoppt, bis das DNA Center nicht mehr gebraucht werden kann. Ein Neustart ist erforderlich.
\textbf{Nachtrag:} Nach mehreren Versuchen mit verschiedenen SSH Servern hat es geklappt. Über die genaue Ursache kann keine Aussage gemacht werden.
}
% Konsequenzen
{Beim Ausfall der Appliance können die Einstellungen nicht wiederhergestellt werden.}
% Workaround
{Keiner}
%Reproduzieren
{
	\begin{enumerate}
		\item Einstellungen $\rightarrow$ System Settings $\rightarrow$ Backup \& Restore
		\item Im Popup \textit{Configure} wählen. 
		\item SSH Servereinstellungen eingeben
		\item \textit{Apply} drücken. 
	\end{enumerate}
}
% Reporter
{Sandro Kaspar}
% Feedback Cisco
{}

\bugreport
% Titel
{Netzwerkgerät OS Update}
% Komponente
{Provision $\rightarrow$ Devices $\rightarrow$ Inventory}
% Dringlichkeit
{Mittel}
% Beschreibung
{Im DNA Center können OS Images der Netzwerkgeräte aktualisiert werden. Dieser Funktion hat bei allen Versuchen immer zu Fehlern geführt und konnte nicht fertiggestellt werden.
\textbf{Wichtig:}
Im Imagerepository muss das Image verfügbar sein. }
% Konsequenzen
{OS Updates müssen manuell durchgeführt werden.}
% Workaround
{Update manuell via TFTP direkt via CLI auf dem Netzwergerät ausführen.}
%Reproduzieren
{
	\begin{enumerate}
		\item Provision $\rightarrow$ Devices $\rightarrow$ Inventory
		\item Gewünschtes Gerät anwählen 
		\item Action $\rightarrow$ Update OS Image
		\item Im Popup Gerät auswählen $\rightarrow$ Update
	\end{enumerate}
}
% Reporter
{Sandro Kaspar}
% Feedback Cisco
{}

\bugreport
% Titel
{DNA Center Update - Appliance nicht nutzbar während Update}
% Komponente
{Einstellungen $\rightarrow$ App Management}
% Dringlichkeit
{Mittel}
% Beschreibung
{Wenn ein \textit{System Update} oder ein \textit{Package Update} durchgeführt wird, kann das GUI des DNA Center nicht verwendet werden. Problematisch: Der Zugriff ist trotzdem möglich. Jedoch sind dann zufällig Funktionen nicht vorhanden oder benutzbar.}
% Konsequenzen
{Appliance nicht nutzbar während Update}
% Workaround
{Während Update GUI nicht verwenden}
%Reproduzieren
{
	Bedingung: Updates sind verfügbar
	\begin{enumerate}
		\item Einstellungen $\rightarrow$ App Management
		\item \textit{System Update} oder \textit{Package Update} wählen
		\item Packages auswählen
		\item Install oder Update wählen
	\end{enumerate}
}
% Reporter
{Philipp Albrecht}
% Feedback Cisco
{Die Empfehlung von Cisco lautet, das DNA Center während Updates nicht zu verwenden.}


\bugreport
% Titel
{Devices mit Namen "NULL" können nicht gelöscht werden}
% Komponente
{Provision $\rightarrow$ Devices $\rightarrow$ Inventory}
% Dringlichkeit
{Niedrig}
% Beschreibung
{Nach der LAN-Automation kommt es vor, dass Devices mit dem Namen "NULL" erscheinen. Diese können nicht über das Action Menü gelöscht werden, da der entsprechende Button deaktiviert ist.}
% Konsequenzen
{Wenn die Synchronisation nicht funktioniert und das Device neu hinzugefügt werden muss, kann es nicht entfernt werden.}
% Workaround
{"NULL-Device" zusammen mit einem funktionierendem Netzwerkgerät markieren. Die \textit{Action} Schaltfläche wird dann klickbar. Das funktionierende Device deselektieren. Die Schaltfläche bleibt danach weiterhin klickbar und das "NULL-Device" kann über \textit{Action} $\rightarrow$ \textit{Delete} gelöscht werden.}
%Reproduzieren
{
	Bedingung: "NULL-Device" in \textit{Inventory} vorhanden
	\begin{enumerate}
		\item Provision $\rightarrow$ Devices $\rightarrow$ Inventory
		\item "NULL-Device" anwählen
		\item Versuchen \textit{Action} Schaltfläche anzuwählen
	\end{enumerate}
}
% Reporter
{Sandro Kaspar}
% Feedback Cisco
{}

\bugreport
% Titel
{https://dnacenter/mypnp \textit{Configurations} nicht löschbar}
% Komponente
{https://dnacenter/mypnp $\rightarrow$ Configurations}
% Dringlichkeit
{Mittel}
% Beschreibung
{Im myPNP können Konfigurationen nicht gelöscht werden.}
% Konsequenzen
{Annahme: Von dort kommen veraltete Informationen die auf die Netzwerkgeräte geschrieben wird.}
% Workaround
{Nicht vorhanden}
%Reproduzieren
{
	Bedingung: In myPNP sind Konfigurationen vorhanden.
	\begin{enumerate}
		\item https://dnacenter/mypnp $\rightarrow$ Configurations
		\item Beliebige Konfiguration anwählen
		\item \textit{Delete} klicken
	\end{enumerate}
}
% Reporter
{Sandro Kaspar}
% Feedback Cisco
{Dieses Feature ist noch in der Beta Phase und offiziell noch nicht verfügbar.}


\bugreport
% Titel
{9xxx Serie Lizenzzuordnung}
% Komponente
{Licence Manager $\rightarrow$ Switches}
% Dringlichkeit
{Mittel}
% Beschreibung
{In der Tabelle \textit{Switch Licence Usage} werden redundante Einträge für Switches der 9xxx Serie angezeigt. Einerseits gibt es den Eintrag \textit{Cisco Catalyst 9300 Series Switches} andererseits \textit{Cisco Catalyst 9xxx Series Switches} Ein Gerät mit der Version 9300 fällt demnach in zwei verschiedene \textit{Model}.}
% Konsequenzen
{Die Lizenzen können nicht zugewiesen werden.}
% Workaround
{Nicht vorhanden}
%Reproduzieren
{
Siehe Beschreibung
}
% Reporter
{Sandro Kaspar}
% Feedback Cisco
{}


\bugreport
% Titel
{Lizenzanzeige}
% Komponente
{Licence Manager $\rightarrow$ Switches}
% Dringlichkeit
{Niedrig}
% Beschreibung
{In der Tabelle \textit{Switch Licence Usage} werden Lizenzen als \textit{Available} angezeigt, die erst in der Zukunft gültig werden.}
% Konsequenzen
{Die Lizenzen können zum aktuellen Zeitpunkt nicht verwendet werden. Die Zahl der \textit{Available Licences} ist nicht korrekt.}
% Workaround
{Nicht vorhanden}
%Reproduzieren
{
	Siehe Beschreibung
}
% Reporter
{Sandro Kaspar}
% Feedback Cisco
{}


\bugreport
% Titel
{PNP}
% Komponente
{Provision $\rightarrow$ LAN Automation}
% Dringlichkeit
{Mittel}
% Beschreibung
{Während der LAN Automation machen die Switches und Router PNP auf das DNA Center. Dies klappt teilweise nicht und der Switch muss zurückgesetzt und neu gestartet werden. Insbesondere wenn mehrere Geräte gleichzeitig aufgesetzt werden sollen funktioniert dies kaum.}
% Konsequenzen
{Die LAN Automation braucht viel manuelle Eingriffe und ist sehr aufwändig.}
% Workaround
{Nicht vorhanden}
%Reproduzieren
{Mehrere Netzwerkgeräte gleichzeitig via PNP in Betrieb nehmen.}
% Reporter
{Sandro Kaspar}
% Feedback Cisco
{}


\bugreport
% Titel
{LAN Automation IP Vergabe}
% Komponente
{Provision $\rightarrow$ Devices $\rightarrow$ Inventory}
% Dringlichkeit
{Mittel}
% Beschreibung
{Wir haben die LAN Automation ein zweites Mal mit einem grösseren IP Pool gestartet. Bei einzelnen Geräten wird aber im DNA Center weiterhin die alte IP angezeigt nachdem diese PNP versucht haben.}
% Konsequenzen
{Die IP Adresse ist ungültig und die Geräte nicht erreichbar.}
% Workaround
{Geräte löschen und Vorgang wiederholen bis die IP Adresse stimmt.}
%Reproduzieren
{
	Siehe Beschreibung
}
% Reporter
{Sandro Kaspar}
% Feedback Cisco
{}

\bugreport
% Titel
{Manuelle Eingriffe Infoblox}
% Komponente
{Design $\rightarrow$ Network Settings $\rightarrow$ IP Address Pool}
% Dringlichkeit
{Niedrig}
% Beschreibung
{Wenn etwas an den IP Pools geändert wird, zum Beispiel das Hinzufügen oder Löschen von IP Pools werden diese Änderungen nicht aktiv, da auf dem Infoblox die entsprechenden Services nicht neu gestartet werden.}
% Konsequenzen
{Die Anzeige im DNA Center zeigt nicht die Realität. DHCP und DNS funktionieren nur eingeschränkt.}
% Workaround
{Die entsprechenden Services auf dem Infoblox Server neu starten}
%Reproduzieren
{
	Siehe Beschreibung
}
% Reporter
{Sandro Kaspar}
% Feedback Cisco
{}

\bugreport
% Titel
{Cisco ISE - TrustSec - Monitor}
% Komponente
{Work Center $\rightarrow$ TrustSec $\rightarrow$ TrustSec Policy $\rightarrow$ Matrix $\rightarrow$ Monitor All - Off}
% Dringlichkeit
{Hoch}
% Beschreibung
{Wird das Monitoring bei den TrustSec Policy aktiviert/deaktiviert, wird diese Änderung nicht immer auf die Switches geschrieben. Ein zusätzlicher Klick auf \textit{Deploy} hilft nicht. Der ISE meldet, es seien bereits alle Änderungen deployed.}
% Konsequenzen
{Monitoring wird nicht aktiviert oder deaktiviert. Das kann bedeuten, dass auf einzelnen Edge Nodes Policies nicht enforced werden.}
% Workaround
{Auf dem Switch die entsprechende Ports down und wieder up nehmen, sodass dieser die Policies vom ISE pulled.}
%Reproduzieren
{
	Siehe Beschreibung
}
% Reporter
{Sandro Kaspar}
% Feedback Cisco
{}


\bugreport
% Titel
{Policies - Contracts - Access Contract}
% Komponente
{Policy $\rightarrow$ Contracts $\rightarrow$ Access Contract}
% Dringlichkeit
{Niedrig}
% Beschreibung
{Wird ein zuvor erstellter Contract bearbeitet, dauert es mehrere Sekunden bis die Rows mit den Actions angezeigt werden.}
% Konsequenzen
{Kann zu Missverständnissen führen, wenn ein Benutzer davon ausgeht, ein Contract sei leer.}
% Workaround
{Warten}
%Reproduzieren
{Unter \textit{Policy $\rightarrow$ Contracts} einen bestehenden Contract bearbeiten.}
% Reporter
{Sandro Kaspar}
% Feedback Cisco
{}
	

\bugreport
% Titel
{Policies - Contracts - Traffic Copy Destination}
% Komponente
{Policy $\rightarrow$ Contracts $\rightarrow$ Traffic Copy Destination $\rightarrow$ Add Traffic Copy Destination}
% Dringlichkeit
{Mittel}
% Beschreibung
{Wird mittels Policy \textit{$\rightarrow$ Contracts $\rightarrow$ Traffic Copy Destination $\rightarrow$ Add Traffic Copy Destination} eine neue Destination hinzugefügt und der \textit{Save} Butten geklickt, speichert dies die Eingabe nicht.}
% Konsequenzen
{Es ist nicht möglich Traffic Copy Destinations anzulegen und somit auch keine Traffic Copy Contracts, da die Destinations dafür benötigt werden.}
% Workaround
{Nicht vorhanden}
%Reproduzieren
{Unter textit{$\rightarrow$ Contracts $\rightarrow$ Traffic Copy Destination $\rightarrow$ Add Traffic Copy Destination} eine neue Destination hinzufügen und speichern.}
% Reporter
{Sandro Kaspar}
% Feedback Cisco
{}







