\section{Bugs}
Im folgenden werden Bugs aufgeführt, die während unserer Arbeit aufgetreten sind. 

\newcommand{\bugreport}[9]{
\subsubsection{#1}
	\begin{table}[H]
		\rowcolors{2}{gray!25}{white}
		\centering
		\begin{tabularx}{\textwidth}{l | X}
			Komponente   & #2       \\
			Dringlichkeit   & #3       \\
			\hline
			Beschreibung   & #4  \\ 
			\hline
			Konsequenzen   & #5  \\ 
			\hline
			Workaround & #6 \\
			\hline
			Reproduzieren & 
			#7
			\\
			\hline
			Reporter  & #8 \\
			Feedback Cisco & #9 \\
		\end{tabularx}
		\caption{Bug: #1}
	\end{table}
}

\bugreport
% Titel
{Backup Server hinzufügen}
% Komponente
{Einstellungen $\rightarrow$ System Settings $\rightarrow$ Backup \& Restore}
% Dringlichkeit
{Hoch}
% Beschreibung
{Nach der Eingabe der Backup Server Einstellungen und den klick auf \textit{Apply} geschieht nichts. Nach einer Weile werden immer mehr Docker Container gestoppt, bis das DNA Center nicht mehr gebraucht werden kann. Ein Neustart ist erforderlich.
\textbf{Nachtrag:} Nach mehreren Versuchen hat es geklappt. Über die genaue Ursache kann keine Aussage gemacht werden. Wir vermuten der Bug wurde in Updates gefixt. 
}
% Konsequenzen
{Beim Ausfall der Appliance können die Einstellungen nicht wiederhergestellt werden}
% Workaround
{Keiner}
%Reproduzieren
{
	\begin{enumerate}
		\item Einstellungen $\rightarrow$ System Settings $\rightarrow$ Backup \& Restore
		\item Im Popup \textit{Configure} wählen. 
		\item SSH Servereinstellungen eingeben
		\item \textit{Apply} drücken. 
	\end{enumerate}
}
% Reporter
{Sandro Kaspar}
% Feedback Cisco
{}

\bugreport
% Titel
{Netzwerkgerät OS Update}
% Komponente
{Provision $\rightarrow$ Devices $\rightarrow$ Inventory}
% Dringlichkeit
{Mittel}
% Beschreibung
{Im DNA Center können OS Images automatisch auf Netzwerkgeräte aufgespielt werden. Dieser Funktion hat bei allen Versuchen immer zu Fehler geführt. 
\textbf{Wichtig:}
Im Imagerepository muss das Image verfügbar sein. }
% Konsequenzen
{OS Updates müssen manuell durchgeführt werden.}
% Workaround
{Update manuell via TFTP direkt via CLI auf dem Netzwergerät ausführen.}
%Reproduzieren
{
	\begin{enumerate}
		\item Provision $\rightarrow$ Devices $\rightarrow$ Inventory
		\item Gewünschtes Gerät anwählen 
		\item Action $\rightarrow$ Update OS Image
		\item Im Popup Gerät auswählen $\rightarrow$ Update
	\end{enumerate}
}
% Reporter
{Sandro Kaspar}
% Feedback Cisco
{}

\bugreport
% Titel
{DNA Center Update - Appliance nicht nutzbar während Update}
% Komponente
{Einstellungen $\rightarrow$ App Management}
% Dringlichkeit
{Mittel}
% Beschreibung
{Wenn ein \textit{System Update} oder ein \textit{Package Update} durchgeführt wird, kann das GUI des DNA Center nicht verwendet werden. Problematisch: Der Zugriff ist trotzdem möglich. Jedoch sind dann zufällig Funktionen nicht vorhanden oder benutzbar.}
% Konsequenzen
{Appliance nicht nutzbar während Update}
% Workaround
{Während Update GUI nicht verwenden.}
%Reproduzieren
{
	Bedingung: Updates sind Verfügbar.
	\begin{enumerate}
		\item Einstellungen $\rightarrow$ App Management
		\item \textit{System Update} oder \textit{Package Update} wählen.
		\item Packages auswählen
		\item Install oder Update wählen.
	\end{enumerate}
}
% Reporter
{Philipp Albrecht}
% Feedback Cisco
{Siehe Workaround}


\bugreport
% Titel
{Devices mit Namen "NULL" können nicht gelöscht werden}
% Komponente
{Provision $\rightarrow$ Devices $\rightarrow$ Inventory}
% Dringlichkeit
{Mittel}
% Beschreibung
{Nach der LAN-Automation kommt es vor, dass Devices mit dem Namen "NULL" erscheinen. Diese können nicht über das Action Menü gelöscht werden.}
% Konsequenzen
{Wenn die Synchronisation nicht funktioniert und das Device neu hinzugefügt werden muss, kann es nicht entfernt werden.}
% Workaround
{"NULL-Device" zusammen mit funktionierendem Netzwerkgerät markieren. Die \textit{Action} Schaltfläche wird dann klickbar. Das funktionierende Device deselektieren. Die Schaltfläche bleibt danach weiterhin klickbar und das "NULL-Device" kann über \textit{Action} $\rightarrow$ \textit{Delete} gelöscht werden.}
%Reproduzieren
{
	Bedingung: "NULL-Device" in \textit{Inventory} vorhanden.
	\begin{enumerate}
		\item Provision $\rightarrow$ Devices $\rightarrow$ Inventory
		\item "NULL-Device" anwählen
		\item Probieren \textit{Action} Schaltfläche anzuwählen
	\end{enumerate}
}
% Reporter
{Sandro Kaspar}
% Feedback Cisco
{}

\bugreport
% Titel
{https://dnacenter/mypnp \textit{Configurations} nicht löschbar}
% Komponente
{https://dnacenter/mypnp $\rightarrow$ Configurations}
% Dringlichkeit
{Mittel}
% Beschreibung
{Im myPNP können Konfigurationen nicht gelöscht werden.}
% Konsequenzen
{Annahme: Von dort kommen veraltete Informationen die auf die Netzwerkgeräte geschrieben wird.}
% Workaround
{Nicht vorhanden.}
%Reproduzieren
{
	Bedingung: In myPNP sind Konfigurationen vorhanden.
	\begin{enumerate}
		\item https://dnacenter/mypnp $\rightarrow$ Configurations
		\item Beliebige Konfiguration anwählen
		\item \textit{Delete} klicken
	\end{enumerate}
}
% Reporter
{Sandro Kaspar}
% Feedback Cisco
{}


\bugreport
% Titel
{9xxx Serie Lizenzzuordnung}
% Komponente
{Licence Manager $\rightarrow$ Switches}
% Dringlichkeit
{Mittel}
% Beschreibung
{In der Tabelle \textit{Switch Licence Usage} werden redudante Einträge für Switches der 9xxx Serie angezeigt. Einerseits gibt es den Eintrag \textit{Cisco Catalyst 9300 Series Switches} andererseits \textit{Cisco Catalyst 9xxx Series Switches} Ein Gerät mit der Version 9300 fällt demnach in zwei Verschiedenen \textit{Model}.}
% Konsequenzen
{Die Lizenzen können nicht zugewiesen werden.}
% Workaround
{Nicht vorhanden.}
%Reproduzieren
{
Siehe Beschreibung
}
% Reporter
{Sandro Kaspar}
% Feedback Cisco
{}



\bugreport
% Titel
{PNP}
% Komponente
{Provision $\rightarrow$ LAN Automation}
% Dringlichkeit
{Mittel}
% Beschreibung
{Während der LAN Automation machen die Switch und Router PNP auf das DNA Center. Dies klappt teilweise nicht und der Switch muss zurückgesetzt und neu gestartet werden.}
% Konsequenzen
{Die LAN Automation braucht viel manuelle Eingriffe und ist sehr aufwändig.}
% Workaround
{Siehe Beschreibung}
%Reproduzieren
{
	Siehe Beschreibung
}
% Reporter
{Sandro Kaspar}
% Feedback Cisco
{}


\bugreport
% Titel
{LAN Automation IP Vergab}
% Komponente
{Provision $\rightarrow$ Devices $\rightarrow$ Inventory}
% Dringlichkeit
{Mittel}
% Beschreibung
{Wir haben die LAN Automation ein zweites Mal mit einem grösseren IP Pool gestartet. Bei einzelnen Geräten wird aber im DNA Center weiterhin die alte IP angezeigt nachdem diese PNP versucht haben.}
% Konsequenzen
{Die IP Adresse ist ungültig und die Geräte nicht erreichbar.}
% Workaround
{Geräte löschen und Vorgang wiederholen bis die IP Adresse stimmt.}
%Reproduzieren
{
	Siehe Beschreibung
}
% Reporter
{Sandro Kaspar}
% Feedback Cisco
{}

\bugreport
% Titel
{Manuelle Eingriffe Infoblox}
% Komponente
{Design $\rightarrow$ Network Settings $\rightarrow$ IP Address Pool}
% Dringlichkeit
{Niedrig}
% Beschreibung
{Wenn etwas an den IP Pools geändert wird (Hinzufügen oder Bearbeiten) werden diese Änderungen nicht aktiv.}
% Konsequenzen
{Beim Adressbereich muss der Infoblox als DHCP Server hinterlegt werden und die Services auf dem Infoblox müssen neu gestartet werden.}
% Workaround
{Siehe Konsequenzen}
%Reproduzieren
{
	Siehe Beschreibung
}
% Reporter
{Sandro Kaspar}
% Feedback Cisco
{}

\bugreport
% Titel
{Cisco ISE - TrustSec - Monitor}
% Komponente
{Work Center $\rightarrow$ TrustSec $\rightarrow$ TrustSec Policy $\rightarrow$ Matrix $\rightarrow$ Monitor All - Off}
% Dringlichkeit
{Niedrig}
% Beschreibung
{Wird das Monitoring bei den TrustSec Policy deaktiviert,  wird diese Änderung nicht auf die Switch geschrieben. Auch beim zusätzlichen Klick auf Deploy geschieht das nicht.}
% Konsequenzen
{Monitoring wird nicht deaktiviert.}
% Workaround
{Auf dem Switch die entsprechende Ports down und wieder up nehmen.}
%Reproduzieren
{
	Siehe Beschreibung
}
% Reporter
{Sandro Kaspar}
% Feedback Cisco
{}






