\section{Problembeschreibung (Stand der Technik)}

\begin{itemize}	
	\item Motivation für die Arbeit, z.B. aus den Schwächen der heutigen Praktiken bzw. Lösungen
	\item Funktionale Anforderungen beschrieben (z.B. als Use Cases (short) mit Aktoren oder in Form von User Stories mit Personas)
	\item Wichtigste NFA/Qualitätsattribute abgedeckt und überprüfbar beschrieben
\end{itemize}

Um den heutigen Anforderungen an Campus Netzwerke im Bezug auf Sicherheit, Wartbarkeit und Skalierbarkeit gerecht zu werden, steht man mit der isolierten Konfiguration einzelner Komponenten schnell vor verschiedenen Problemen. In erster Linie ist es extrem Aufwändig alle Konfigurationen manuell zu erstellen. Selbst das Hinzufügen von einfachen Richtlinien oder zum Beispiel neuen Firmenabteilungen können zu gewaltigem Aufwand führen. Desweiteren verliert man schnell die Übersicht und ist gezwungen umfangreiche Dokumentationen zu erstellen.

Deshalb sind Lösungen gefragt, um diese Verwaltung übersichtlich und zentral zu lösen. 



