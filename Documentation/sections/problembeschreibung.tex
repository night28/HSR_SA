\section{Problembeschreibung}
Wer den heutigen Anforderungen der Campus Netzwerke in Bezug auf Sicherheit, Wartbarkeit und Skalierbarkeit gerecht werden möchte, steht mit der isolierten Konfiguration einzelner Komponenten schnell vor verschiedenen Problemen. In erster Linie ist es extrem aufwändig, alle Konfigurationen manuell zu erstellen. Selbst das Hinzufügen von einfachen Richtlinien oder neuen Firmenabteilungen kann zu gewaltigem Aufwand führen. Des Weiteren kann die Übersicht schnell verloren werden und eine umfangreiche Dokumentation zu erstellen ist unumgänglich. Häufig kommen selbstgeschriebene Scripts, zum Beispiel mithilfe von NAPALM \cite{napalm} zur automatisierten Konfiguration zum Einsatz. Für das Monitoring des Netzwerkes sind zusätzlich Tools wie icinga2 \cite{icinga2} oder ähnliches nötig. 

~\\
Typische Herausforderungen bei den klassischen Campus Netzwerken:
\begin{itemize}
	\item Zu wenig VLANs
	\item Mobilität von Endgeräten
	\item Mobilität von Benutzern
	\item Durchsetzen von Sicherheitsregeln mithilfe von Firewalls
	\item Direkte Abhängigkeit von Berechtigungen und IP Subnetzen
	\item Mehrere unabhängige Tools mit Informationsredundanz
	\item Komplexe Fehlersuche über verschiedene Komponenten/Geräte hinweg
\end{itemize}

~\\
Genau hier setzt das Cisco DNA Center an. Es fasst alle diese Tools unter einem Dach zusammen und bietet eine übergreifende Plattform. 

