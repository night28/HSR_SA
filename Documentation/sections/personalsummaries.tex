\section{Persönliche Summaries}
\subsection{Sandro Kaspar}

\subsection{Philipp Albrecht}
Mit der Vorstellung wie klassische Netzwerke konfiguriert werden, bin ich an das DNA Center mit grossen Erwartungen gestossen. Network Orchestration mit zentralen Kontrollern habe ich bisher nur von Ubiquiti und Cisco Meraki gekannt. Als wir nach langen warten endlich die Hardware mitte April bekommen haben, merkte ich, dass meine Erwartungen viel zu hoch waren. Während ich mir wie bei Cisco Meraki eine einfache intuitive "Clicki-Bunti" Lösung vorgestellt habe, stiess ich an ein unintuitives Etwas, komplizierten Lizenzen und haufenweise Bugs. Alle Operationen und Versuche waren geprägt vom langen warten bis irgendwelche Geräte ihren Reboot durchgeführt haben und durchstöbern von als Marketingunterlagen strukturierte Bedienungsanleitungen. Schnell merkte ich zwei Dinge. Einerseits den Mangel an Erfahrungen und Wissen mit Cisco ISE, LISP, VXLAN und andererseits, dass das effektive Erlebnis mit dem DNA Center weit abweicht von den farb-freudigen Marketing Videos auf der Webseite von Cisco. 
Im persönlichen Zeitmanagement kam mit dem späten Eintreffen der nicht funktionierend Appliance noch ein weiteren Problem. Seit beginn der Arbeit waren nun schon fast zwei Monate vergangen und plötzlich musste ich viel mehr Zeit in die Semesterarbeit investieren. Da ich Teilzeit studiere, nebenbei Arbeite und jeweils von Zürich nach Rapperswil pendle, konnte ich nicht einfach plötzlich mehr Zeit für die Semesterarbeit investieren. 
Alles in allem fand ich unsere Arbeit sehr spannend. Das Ergebnis hingegen ist sehr ernüchternd und nicht zufriedenstellend. Das DNA Center ist nicht wie Erwartet ein fertiges ausgereiftes Produkt, sondern eine riesige Baustelle.

\subsection{Jessica Kalberer}
Das Themengebiet Network Design and Seurity hat mich schon seit Anfang des Studiums interessiert und mich nun in der Studienarbeit vor neue Herausforderungen gestellt. Als ich zum ersten mal von Cisco DNA Center hörte, war ich fasziniert von der ganzen Appliance. Der Gedanke das nun alles über eine einzige Appliance konfiguriert und verwaltet werden konnte, war einfach traumhaft. Am Anfang des Projektes musste ich mich einige Stunden in die  Technologien einlesen, da vieles für mich neu war. Bisher kannte ich nur die traditionellen Netzwerk Design die aus einem Acces, Distribution und Core Layer bestanden. Leider verschob sich unsere ganze Arbeit etwas, da die Hardware relativ spät bei uns ankam. Mitte April konnten wir dann mit der ganzen Installation und Konfiguration starten. Die  Konfiguration des DNA Center war relativ Ernüchternd, da vieles noch nicht fehlerfrei funktionierte und darum manuell konfiguriert werden musste. \\
Die Arbeit im dreier Team empfand ich als angenehm. Es war jedoch teilweise etwas schwierig, wenn Konfigurationen im DNA Center gemacht wurden und nicht alle in einem Raum sassen, so das nicht jeder wusste was gerade gemacht wird. Da das DNA Center fehleranfällig war, musste immer genau abgesprochen werden, wer was Konfiguriert und wann etwas neu gestartet wird. Teilweise funktionierten Ansichten nicht mehr genau oder der ISE wurde wahllos nicht mehr angezeigt. Die Arbeit im Team hatte aber zum Vorteil, das viele Probleme besprochen werden konnten und fast immer jemand wusste wie man es anders probieren könnte.