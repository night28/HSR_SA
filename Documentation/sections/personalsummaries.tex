\section{Persönliche Summaries}
\subsection{Sandro Kaspar}

\subsection{Philipp Albrecht}
Mit der Vorstellung wie klassische Netzwerke konfiguriert werden, bin ich an das DNA Center mit grossen Erwartungen gestossen. Network Orchestration mit zentralen Kontrollern habe ich bisher nur von Ubiquiti und Cisco Meraki gekannt. Als wir nach langen warten endlich die Hardware mitte April bekommen haben, merkte ich, dass meine Erwartungen viel zu hoch waren. Während ich mir wie bei Cisco Meraki eine einfache intuitive "Clicki-Bunti" Lösung vorgestellt habe, stiess ich an ein unintuitives Etwas, komplizierten Lizenzen und haufenweise Bugs. Alle Operationen und Versuche waren geprägt vom langen warten bis irgendwelche Geräte ihren Reboot durchgeführt haben und durchstöbern von als Marketingunterlagen strukturierte Bedienungsanleitungen. Schnell merkte ich zwei Dinge. Einerseits den Mangel an Erfahrungen und Wissen mit Cisco ISE, LISP, VXLAN und andererseits, dass das effektive Erlebnis mit dem DNA Center weit abweicht von den farb-freudigen Marketing Videos auf der Webseite von Cisco. 
Im persönlichen Zeitmanagement kam mit dem späten Eintreffen der nicht funktionierend Appliance noch ein weiteren Problem. Seit beginn der Arbeit waren nun schon fast zwei Monate vergangen und plötzlich musste ich viel mehr Zeit in die Semesterarbeit investieren. Da ich Teilzeit studiere, nebenbei Arbeite und jeweils von Zürich nach Rapperswil pendle, konnte ich nicht einfach plötzlich mehr Zeit für die Semesterarbeit investieren. 
Alles in allem fand ich unsere Arbeit sehr spannend. Das Ergebnis hingegen ist sehr ernüchternd und nicht zufriedenstellend. Das DNA Center ist nicht wie Erwartet ein fertiges ausgereiftes Produkt, sondern eine riesige Baustelle.

\subsection{Jessica Kalberer}