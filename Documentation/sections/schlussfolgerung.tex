\section{Schlussfolgerungen}

\subsection{Erreichte Ziele}
Zusätzlich zum Ziel das Netzwerk mit Underlay und Overlay aufzusetzen, konnten alle Use Cases praktisch oder theoretisch in der Arbeit abgedeckt werden. 
\subsection{Mögliche Verbesserungen}
Um noch mehr Netzwerkorchestrierung über Programmierschnittstellen zu erreichen sind die kommenden Erweiterungen der API des Cisco DNA Centers unabdingbar. Die in dieser Arbeit bisher aufgetretenen Bugs, sind für eine neue und moderne Software nicht ungewöhnlich, müssen in zukünftigen Releases aber behoben werden.
 
\subsection{Zukunft}
Die Verwaltung des Netzwerkes durch einen zentralen Controller, wie das Cisco DNA Center, schafft viel Vorteile. In der Zukunft können sich Netzwerkingenieure mehr auf ertragreiche Aufgaben, wie das Netzwerkdesign und die Überwachung, konzentrieren. Dank des vereinfachten und zentralen Troubleshootings mit der Cisco DNA Center Assurance sind Netzwerkprobleme im Handumdrehen gelöst. Des Weiteren bietet die API in den kommenden Versionen des Cisco DNA Centers die Möglichkeit alle Netzwerkgeräte über eine einzige Schnittstelle zu programmieren und an externe Systeme anzubinden. Der komplexe und aufwändige Zugriff auf jedes einzelne Gerät entfällt. 