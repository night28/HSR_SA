\section{Schlussfolgerungen}
Zusammenfassung und Ausblick
\subsection{Erreichte Ziele}
Nebst dem Ziel das Netzwerk mit Underlay und Overlay aufzusetzen, konnten Alle Usecases praktisch oder theoretisch in der Arbeit abgedeckt werden. 
\subsection{Mögliche Verbesserungen}
Um noch mehr Netzwerkorchestrierung über Programmierschnittstellen zu erreichen sind kommende Erweiterungen der API des Cisco DNA Centers unabdingbar. Die in unserer Arbeit bisher aufgetretenen Bugs gehören zu jeder Software dazu und werden sicherlich bei kommenden Versionen behoben. 
 
\subsection{Zukunft}
Die zentrale Verwaltung des Netzwerkes durch einen Controller wie das Cisco DNA Center schafft viel Vorteile. In der Zukunft können sich Netzwerkingenieure mehr auf ertragreiche Aufgaben wie das Netzwerkdesign und die Überwachung konzentrieren. Dank des vereinfachten und zentralen Troubleshooting mit dem Cisco DNA Center Assurance sind Netzwerkprobleme im Handumdrehen gelöst. Des weiteren bietet die API in den kommenden Versionen des Cisco DNA Centers die Möglichkeit alle Netzwerkgeräte über eine einzige Schnittstelle zu programmieren. Der komplexe und mühsame Zugriff auf jedes einzelne Gerät entfällt. 