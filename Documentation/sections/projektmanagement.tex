\section{Projektmanagement}

\subsection{Projektübersicht}
Das Hauptziel dieser Studienarbeit ist die Installation des DNA Centers und Integration eines Campus Labor Netzwerkes.

\subsubsection{Ziele der Projektes}
Da SDN im Campus Bereich Neuland ist, soll die SDA Lösung vom Hersteller Cisco ausgearbeitet werden. Dazu gehören folgende Ziele:

\begin{itemize}
	\item Installation von DNA Center und Integration vom Campus Labor Netzwerk
	\item Definition von Benutzer- und Geräteprofilen, um basierend auf Geschäftsanforderungen die Zugriffsrechte und Netzwerksegmentierung zu verwalten und so das Netzwerk sicher zu halten
	\item Verwendung von Erkenntnissen von DNA Analytics and Assurance für eine proaktive Überwachung, Fehlerbehebung und Optimierung des Netzwerks
	\item Integration vom bestehendem IPAM Tool im DNA Center
	\item Erstellung von wöchentlichen Reports über den Campus Netzwerk Status in einem E-Mail 
\end{itemize}

\subsection{Projektorganisation}
Diese Studienarbeit wird von drei Personen umgesetzt und durch zwei Betreuer überwacht.

\subsubsection{Organisationsstruktur}
\begin{figure}[H]
	\centering
	\includegraphics[height=5cm]{img/Organisationsstruktur.png}
	\caption{Organisationsstruktur}
	\label{fig:Organisationsstruktur}
\end{figure}

\subsection{Management Abläufe}
Für die Umsetzung der Studienarbeit stehen insgesamt 15 Wochen und pro Person 240 Stunden zur Verfügung. In einer Woche liegt das Arbeitspensum von 16 Stunden pro Person vor. Das Projekt startet am 19. Februar 2018 und endet am 15. Juni 2018.

\subsubsection{Zeitliche Planung}
Die zeitliche Planung, sowie die Verwaltung der Arbeitspakete erfolgte auf Waffle.io. Die Planung wird während dem Projekt laufend aktualisiert und angepasst. Die Arbeitszeiten werden während der Arbeitsausführung mit Toggle erfasst.

\begin{figure}[H]
	\centering
	\includegraphics[width=16cm]{img/ZeitlichePlanung_v4.png}
	\caption{Projektplanung}
	\label{fig:Projektplanung}
\end{figure} 

\subsubsection{Meilensteine}
Folgende Meilensteine sind für das Projekt definiert:
\begin{table}[H]
	\rowcolors{2}{gray!25}{white}
	\centering
	\begin{tabularx}{\textwidth}{p{1cm}| p{2.5cm}| X}
		\rowcolor{gray!50}
		\textbf{Nr} & \textbf{Datum} & \textbf{Meilenstein} \\
		\hline	
		M0 & 27.02.2018 & Kickoff Meeting \\
		M1 & 10.04.2018 & Projektplanung abgeschlossen \\
		M2 & 24.04.2018 & Inbetriebnahme Hardware abgeschlossen \\
		& 16.05.2018 & Zwischenpräsentation \\
		M3 & 01.06.2018 & Fabric Konfiguration \\
		M4 & 10.06.2018 & Definierung von Benutzer- und Geräteprofilen \\
		M5 & 12.06.2018 & Reporting und Monitoring \\
		M6 & 13.06.2018 & Freigabe des Abstracts \\
		M7 & 13.06.2018 & Abgabe Projekt \\
		& 15.06.2018 & Endpräsentation \\
	\end{tabularx}
	\caption{Meilensteine}
	\label{tab:Meilensteine}
\end{table}


\subsubsection{Arbeitspakete}
Alle Arbeitspakete werden in Waffle.io erfasst und sind unter folgendem Link ersichtlich:
\href{Waffle.io}{https://waffle.io/night28/HSR\_SA}
\subsubsection{Besprechungen}
Die Besprechungen mit dem Betreuer finden an den nachfolgend aufgelisteten Tagen statt:
\begin{itemize}
	\item jeden Dienstag zwischen 15.10 - 16.10 Uhr
\end{itemize}

Offene Traktanden und Probleme werden mit dem Betreuer diskutiert. Nach dieser Besprechung wird jeweils in einem Team-Meeting das weitere Vorgehen geplant.


\subsection{Infrastruktur}
Die Organisation der Arbeit und Teammitglieder wird durch folgende Werkzeuge unterstützt:

\begin{figure}[H]
	\centering
	\includegraphics[width=13cm]{img/EingesetzteToolsZurOrganisation.png}
	\caption{Übersicht über die Verknüpfung der eingesetzten Werkzeuge zur internen Organisation.}
	\label{fig:Interne Organisationsstruktur}
\end{figure} 

Unsere Tools sind unter folgenden Links einsehbar:
\paragraph{GitHub} \href{https://github.com/night28/HSR_SA}{https://github.com/night28/HSR\_SA} 

\paragraph{Waffle.io} \href{https://waffle.io/night28/HSR\_SA}{https://waffle.io/night28/HSR\_SA}

\paragraph{Toggl} \href{https://toggl.com/}{https://toggl.com/}
 
\subsection{Risiko Management}

\subsubsection{Umgang mit Risiken}

Risiken lassen sich nicht vermeiden. Aus diesem Grund sind nachfolgend mögliche Risiken aufgeführt. Des Weiteren wurden vorbeugende Massnahmen definiert, um die Eintrittswahrscheinlichkeit von Risiken mit schwerwiegenden Konsequenzen zu reduzieren. Für den Fall, dass ein Risiko dennoch eintreten sollte, sind entsprechende Massnahmen definiert, um den Schaden möglichst gering zu halten.
Sollten sich während dem Projekt neue potenzielle Risiken zeigen, wird dieses Dokument laufend aktualisiert.

\begin{landscape}

\subsubsection{Risiken}
\newcommand*\rot{\rotatebox{90}}
\rowcolors{2}{gray!25}{white}
\begin{longtable}{|m{0.5cm}|m{3cm}|m{5cm}|m{0.75cm}|m{0.75cm}|m{0.75cm}|m{5cm}|m{5cm}|} 
	\hline
	\rot{Nummer} & \rot{Titel} & \rot{Beschreibung} & \rot{\shortstack[l]{maximaler\\Schaden [h]}} & \rot{\shortstack[l]{Eintritts-\\wahrscheinlichkeit}} & \rot{\shortstack[l]{Gewichteter\\Schaden [h]}} & \rot{Vorbeugung} & \rot{\shortstack[l]{Verhalten beim\\Eintreten}} \\
	\hline\hline
	1 & Ausfall eines Teammitglieds & Ausfall auf Grund unvorhergesehener Ereignisse wie Krankheit, Unfall etc. & 40 & 15\% & 6 & Reserven einplanen, Kommunikation sicherstellen, damit andere Teammitglieder die Aufgaben übernehmen können & Tasks des ausgefallen Mitglieds möglichst auf die anderen Teammitglieder aufteilen. \\ 
	\hline
	2 & Hardwareausfall DNA-Center & DNA-Center Appliance fällt durch Hardwaredefekt aus & 30 & 5\% & 1.5 & keine vorbeugenden Massnahmen möglich & Austausch im Rahmen der Garantie veranlassen \\
	\hline
	3 & Fehlendes Know-How & Da viele der Themen neu sind, kann entsprechendes Wissen fehlen & 40 & 20\% & 8 & Zeit einplanen, um sich in neue Themen einzuarbeiten & Fehlendes Wissen sobald wie möglich aneignen. Bei Bedarf Rat der Betreuer einholen \\
	\hline
	4 & Konflikte oder Missverständnisse im Team & Das Team ist sich bezüglich wichtigen Entscheidungen uneinig & 25 & 15\% & 3.75 & Entscheidungen stets mit Begründung dokumentieren & Kann auch mit Hilfe der Dokumentation keine Einigung gefunden werden, fachlichen Rat des Betreuers einholen \\
	\hline
	5 & Missverständnisse im Team & Im Team herrscht Uneinigkeit über bereits getroffene Entscheidungen & 20 & 20\% & 4 & Protokolle führen und Entscheidungen klar dokumentieren & Protokolle und Dokumentationen beiziehen \\
	\hline
	6 & Ausfall Server / Netzwerkinfrastruktur & Ausfall der von der HSR zur Verfügung gestellten Infrastrukturkomponenten & 30 & 10\% & 3 & Keine Vorbeugenden Massnahmen möglich & Sobald die Infrastruktur wieder verfügbar ist, Systeme erneut in Betrieb nehmen \\
	\hline
	7 & Lieferverzögerung Hardware & Die von Cisco bestellte Hardware kommt später als angekündigt & 30 & 18\% & 5.4 & Keine Vorbeugenden Massnahmen möglich & Projektplanung an neue Gegebenheiten anpassen, notfalls Projektumfang in Absprache mit Betreuer anpassen \\
	\hline
	8 & Zeitaufwände falsch geschätzt & Auf Grund falscher Schätzungen kommt es zu Verzögerungen im Projekt & 30 & 25\% & 7.5 & Laufende Kontrolle des Projektfortschritts um Probleme frühzeitig zu erkennen, Reserven einplanen & Verbleibende Schätzungen korrigieren, Planung anpassen \\
	\hline
	9 & Datenverlust & Verlust von projektbezogenen Daten wie Dokumentationen, Konfigurationen etc. & 40 & 5\% & 2 & Regelmässige und verteilte Backups aller Daten erstellen & Verlorenen Daten aus Backups wiederherstellen, fehlende Daten neu erarbeiten \\
	\hline
	10 & Unausgereifte Software & Verzögerung des Projektes durch unvorhergesehene Hürden, da Software nicht genügend auf Funktionalität getestet und Dokumentiert. Software steht noch in einem frühen Release. & 80 & 5\% & 4 & Über aktuelle Funktionalitäten und Bugs informieren & Bugs reporten und bei Möglichkeit diese umgehen. Falls nötig Unterstützung beim Hersteller suchen. \\
	\hline
\end{longtable}

\end{landscape}

\includepdf{pdfincludes/risikograph}

\subsubsection{Eingetretene Risiken}
Nachfolgend werden die eingetretenen Risiken genauer erläutert.
\paragraph{Lieferverzögerung Hardware}
~\\
Leider wurde die Hardware nicht zum geplanten Zeitpunkt geliefert. Deshalb wurde die Projektplanung an die neuen Gegebenheiten angepasst. \\
Nachfolgend die alte Projektplanung:
\begin{figure}[H]
	\centering
	\includegraphics[height=5cm]{img/ZeitlichePlanung_v1.png}
	\caption{alte Projektplanung}
	\label{fig:alte Projektplanung}
\end{figure} 

Folgende Meilensteine waren für das Projekt definiert:
\begin{table}[H]
	\rowcolors{2}{gray!25}{white}
	\centering
	\begin{tabularx}{\textwidth}{p{1cm}| p{2.5cm}| X}
		\rowcolor{gray!50}
		\textbf{Nr} & \textbf{Datum} & \textbf{Meilenstein} \\
		\hline	
		M0 & 27.02.2018 & Kickoff Meeting \\
		M1 & 20.03.2018 & Projektplanung abgeschlossen \\
		M2 & 03.04.2018 & Inbetriebnahme Hardware abgeschlossen \\
		M3 & 17.04.2018 & Definierung von Benutzer- und Geräteprofilen \\
		M4 & 01.05.2018 & Integration von bestehenden IPAM in DNA Center \\
		M5 & 15.05.2018 & Reporting \& Monitoring \\
		M6 & 28.05.2018 & Freigabe des Abstracts \\
		M7 & 01.06.2018 & Abgabe Projekt \\
	\end{tabularx}
	\caption{alte Meilensteine}
	\label{tab:alte Meilensteine}
\end{table}

Die neue Projektplanung sieht nun folgendermassen aus:
\begin{figure}[H]
	\centering
	\includegraphics[height=5cm]{img/ZeitlichePlanung_v3.png}
	\caption{neue Projektplanung}
	\label{fig:neue Projektplanung}
\end{figure} 


Folgende Meilensteine sind nun aufgrund der Lieferverzögerung für das Projekt definiert:
\begin{table}[H]
	\rowcolors{2}{gray!25}{white}
	\centering
	\begin{tabularx}{\textwidth}{p{1cm}| p{2.5cm}| X}
		\rowcolor{gray!50}
		\textbf{Nr} & \textbf{Datum} & \textbf{Meilenstein} \\
		\hline	
		M0 & 27.02.2018 & Kickoff Meeting \\
		M1 & 10.04.2018 & Projektplanung abgeschlossen \\
		M2 & 17.04.2018 & Inbetriebnahme Hardware abgeschlossen \\
		M3 & 24.04.2018 & Definierung von Benutzer- und Geräteprofilen \\
		M4 & 08.05.2018 & Integration von bestehenden IPAM in DNA Center \\
		M5 & 22.05.2018 & Reporting \& Monitoring \\
		M6 & 28.05.2018 & Freigabe des Abstracts \\
		M7 & 01.06.2018 & Abgabe Projekt \\
	\end{tabularx}
	\caption{neue Meilensteine}
	\label{tab:neue Meilensteine}
\end{table}

\paragraph{Unausgereifte Software und fehlendes Know-How}
~\\
Das DNA Center befand sich beim Beginn unserer Studienarbeit noch in der Version 1.1.3. Bis zur Abgabe wurde die Version 1.1.6 veröffentlicht, auf welche wir unser DNA Center auch aktualisiert hatten. 
\begin{figure}[H]
	\centering
	\includegraphics[height=6cm]{img/ReleaseNotes.png}
	\caption{Release Notes}
	\label{fig:Release Notes}
\end{figure}
Das DNA Center enthält in diesen frühen Versionen noch viele Bugs und auch Beta Features, welche oft zu Problemen führen können. Die Funktionalitäten sind teilweise nur beschränkt so umsetzbar, wie sie angekündigt und beschrieben wurden. 
Bei unserem ersten Versuch mit der Version 1.1.3 stiessen wir auf das Problem, dass wir die Geräte über die LAN Automation nicht in Betrieb nehmen konnten, da nicht einmal ein DHCP Server auf dem Seed Device konfiguriert wurde. Weitere Probleme kamen auch beim Provisionierungsprozess hinzu. Geräte welche vorher verwaltet werden konnten, waren auf einmal nicht mehr erreichbar im DNA Center, obwohl dies manuell per SSH kein Problem darstellte. Ein Versuch das DNA Center per Backup zu sichern, brachte das ganze DNA Center zum Absturz. Nachdem viele solche Hürden und Probleme aufgetaucht waren, entschieden wir uns es mit einem Out of Band Management zu versuchen. Hierzu musste der Konfigurations-Wizard des DNA Centers nochmal gestartet werden, um das zweite Netzwerkinterface zu definieren. Das erneute Durchführen dieses Konfigurations-Wizard führte zum kompletten Absturz, so dass die ganze DNA Center Appliance gar nicht mehr startete. \\
Nach einer zweiten Installation des DNA Centers versuchten wir erneut die LAN Automation auszuführen, um ein Underlay bereitzustellen. Diesmal funktionierte das Hinzufügen eines Seed-Devices. Die LAN Automation soll nach der Konfiguration eines Seed-Devices so oft wie nötig gestartet und gestoppt werden können. Sollte später ein weiterer Switch hinzu kommen, so könnte diese erneut für dieses Device gestartet werden. In unserem Fall führe dies zu Problemen mit der Konfiguration des IS-IS Protokolls. Es wurden nur einzelne Point to Point Interfaces konfiguriert. Aus diesem Grund war für einzelne Geräte keine Kommunikation zum DNA Center möglich. 
Dies führte bei einigen Konfigurationen zu Verwirrung, da wir teilweise nicht verifizieren konnten, ob es sich um ein falsches Verhalten der Software oder einen Fehler unsererseits handelte. Aus diesem Grund wurde beschlossen, uns für einen Tag einen Cisco Experten zur Verfügung zu stellen. Wir konnten mit ihm die Konfiguration noch einmal von Grund auf durchführen und kamen bis zur Konfiguration eines Seed-Devices für die LAN Automation. An diesem Punkt stiessen wir aber wieder auf diverse Hindernisse, bei welchen uns auch der Cisco Experte zu dieser Zeit nicht weiterhelfen konnte. Nach eigenen weiteren Versuchen gelang es uns jedoch das Problem zu beheben und die LAN Automation auf einem weiteren Gerät durchzuführen.\\
Des Weiteren fehlen Dokumentationen zu der Verwendung von Policies oder der genauen Verwendung der \textit{Authentication Templates} für das \textit{Host Onboarding}. Zur Bedeutung der verschiedenen Authentication Templates konnte uns jedoch der Experte von Cisco Auskunft geben. 

Da bei uns bereits zwei eher schwerwiegende Risiken eingetreten waren, wurde entschieden, dass der Abgabetermin um knapp zwei Wochen, auf den 13. Juni 2018 verschoben wird. Das hatte folgende Anpassungen in der Projektplanung zur Folge:


\begin{figure}[H]
	\centering
	\includegraphics[width=16cm]{img/ZeitlichePlanung_v4.png}
	\caption{Erweiterte Anpassung der Projektplanung}
	\label{fig:Erweiterte Anpassungen der Projektplanung}
\end{figure} 


Folgende Meilensteine sind nun auf Grund der Lieferverzögerung für das Projekt definiert:
\begin{table}[H]
	\rowcolors{2}{gray!25}{white}
	\centering
	\begin{tabularx}{\textwidth}{p{1cm}| p{2.5cm}| X}
		\rowcolor{gray!50}
		\textbf{Nr} & \textbf{Datum} & \textbf{Meilenstein} \\
		\hline	
		M0 & 27.02.2018 & Kickoff Meeting \\
		M1 & 10.04.2018 & Projektplanung abgeschlossen \\
		M2 & 24.04.2018 & Inbetriebnahme Hardware abgeschlossen \\
		   & 16.05.2018 & Zwischenpräsentation \\
		M3 & 01.06.2018 & Fabric Konfiguration \\
		M4 & 10.06.2018 & Definierung von Benutzer- und Geräteprofilen \\
		M5 & 12.06.2018 & Reporting und Monitoring \\
		M6 & 13.06.2018 & Freigabe des Abstracts \\
		M7 & 13.06.2018 & Abgabe Projekt \\
		   & 15.06.2018 & Endpräsentation \\
	\end{tabularx}
	\caption{Erweiterte Anpassung der Meilensteine}
	\label{tab:Erweiterte Anpassung der Meilensteine}
\end{table}

In der Grafik ist ersichtlich, dass die komplette Konfiguration des DNA Centers nach der Zwischenpräsentation stattfand. Geplant war die Fabric Konfiguration schon in der zehnten Woche, jedoch funktionierte zu diesem Zeitpunkt die LAN Automation nicht und die Geräte wurden manuell zum DNA Center hinzugefügt, sodass eine Fabric konfiguriert werden konnte. Kurz vor der Zwischenpräsentation war die Definierung der Benutzer- und Geräteprofile wichtig, da an der Präsentation unter Anderem die Konnektivität zwischen zwei Clients vorgeführt werden sollte. Dies war jedoch wegen mehreren aufgetretenen Fehlern und Problemen nicht möglich. Der Versuch das DNA Center nochmals komplett mit einem Out of Band Management zu konfigurieren scheiterte leider. Der Maglev Configuration Wizard brach am Schluss der Konfigurationen mit einem Fehler ab und brachte das ganze DNA Center in einen "not bootable" Zustand. \\
Dies war ein guter Zeitpunkt um die komplette Installation des DNA Center von vorne zu beginnen. Durch die vielen aufgetretenen Probleme wurde uns, wie schon oben erwähnt, für einen Tag ein Experte von Cisco zur Seite gestellt. Mit ihm konnten wir die Konfiguration des Underlay Netzwerkes bis zum Definieren eines ersten Seed-Devices durchführen.
