\section{Benutzerhandbuch}

\subsection{Updates}
Es ist wichtig das DNA Center auf einem aktuellen Stand zu halten, da sehr häufig neue Updates publiziert werden.

Auf folgender Webseite veröffentlicht Cisco die Sicherheitslücken und erklärt gleich zu welcher Version das DNA Center geupdatet werden muss:
\url{https://tools.cisco.com/security/center/publicationListing.x?resourceIDs=233151\&apply=1\&totalbox=1\&pt0=Cisco\&cp0=233151\#~FilterByProduct}

\subsubsection{Updates installieren}
\begin{enumerate}
	\item Ausgehend vom DNA Center Dashboard nach \textit{Settings $\rightarrow$ System Settings $\rightarrow$ App Management} navigieren
	\item Das gewünschte System Update oder Package markieren und mittels \textit{Action $\rightarrow$ Download} herunterladen
	\item Das heruntergeladene Update markieren und mittels \textit{Action $\rightarrow$ Install} installieren
\end{enumerate}

Es ist zu beachten, dass System Updates immer vor den Package Updates installiert werden müssen. Des Weiteren sollen die restlichen Funktionen des DNA Centers während dem Update nicht verwendet werden.

\subsection{Design}
Mit dem Design wird die physische Struktur bis auf die Gebäude genau hinterlegt. Zusätzlich werden Informationen hinterlegt, die das DNA Center während der Provisionierung auf die Netzwerkkomponenten schreibt.

\subsubsection{Site hinzufügen}
\begin{enumerate}
	\item Ausgehend vom DNA Center Dashboard nach \textit{Design $\rightarrow$ Network Hierarchy} navigieren
	\item \textit{Add Site} wählen
	\item Im neuen Popup den gewünschten Namen eingeben
	\item \textit{Add} anwählen
\end{enumerate}

\subsubsection{Gebäude zur Site hinzufügen}
\begin{enumerate}
	\item Ausgehend vom DNA Center Dashboard nach \textit{Design $\rightarrow$ Network Hierarchy} navigieren
	\item \textit{Add Site} wählen
	\item Im neuen Popup den gewünschten Namen eingeben
	\item \textit{Building} anwählen
	\item Adresse und/oder Koordinaten eingeben
	\item \textit{Add} anwählen.
\end{enumerate}


\subsubsection{Netzwerkdienste Konfigurieren}
\begin{enumerate}
	\item Ausgehend vom DNA Center Dashboard nach \textit{Design $\rightarrow$ Network Settings} navigieren. 
	\item \textit{Global} wählen
	\item Bei \textit{AAA Server Network} und \textit{Client/Endpoint} anwählen
	\item Bei \textit{Network $\rightarrow$ ISE} wählen und entsprechende IP Adresse eingeben
	\item Bei \textit{Network $\rightarrow$ RADIUS} wählen und entsprechende IP Adresse eingeben
	\item Bei \textit{Client/Endpoint $\rightarrow$ Servers $\rightarrow$ ISE} wählen und entsprechende IP Adresse eingeben
	\item Bei \textit{Client/Endpoint $\rightarrow$ Protocol $\rightarrow$ RADIUS} wählen und entsprechende IP Adresse eingeben
	\item Bei \textit{DHCP Server} das \textbf{PLUS}-Zeichen anklicken und die IP Adresse des DHCP Servers hinterlegen 
	\item \textit{SYSLOG Server}, \textit{SNMP Server} und \textit{Netflow Collector Server} können leer gelassen werden
	\item Mit einem Klick auf \textit{Save} ist alles abzuspeichern
\end{enumerate}

\subsubsection{Device Credentials hinterlegen}
\begin{enumerate}
	\item Ausgehend vom DNA Center Dashboard nach \textit{Design $\rightarrow$ Network Settings $\rightarrow$ Device Credentials} navigieren
	\item \textit{Add} wählen
	\item \textit{Name}, \textit{Username}, \textit{Password} und \textit{Enable Password} eingeben
	\item Mit einem Klick auf \textit{Save} alles abspeichern
\end{enumerate}

\subsubsection{IP Address Pools hinzufügen}
Das DNA Center benötigt verschiedene IP Adressen Pools. Die Grösse der Pools ist entsprechend der Anforderungen in der Umgebung zu wählen.
\begin{itemize}
	\item Ein Pool für die LAN Automation (P2P Links, Loopback Adressen)
	\item Ein Pool für die Border Konfiguration
	\item Für jedes VN ein Pool
\end{itemize}

Weitere Pools können jederzeit hinzugefügt werden.

\begin{enumerate}
	\item Ausgehend vom DNA Center Dashboard nach \textit{Design $\rightarrow$ Network Settings $\rightarrow$ IP Address Pools} navigieren
	\item \textit{Add IP Pool} wählen
	\item Nun gilt es einen IP Pool Name, ein IP Subnet, ein CIDR Präfix, eine Gateway IP Adresse einzugeben
	\item Der korrekte DHCP Server und DNS Server ist per Dropdown auszuwählen
	\item Mit einem Klick auf \textit{Save} die Eingaben speichern
\end{enumerate}

\subsubsection{Templates erstellen}
Um Konfigurationen auf den Geräten vorzunehmen, die nicht vom DNA Center abgedeckt sind, können Templates definiert werden.
\begin{enumerate}
	\item Ausgehend vom DNA Center Dashboard nach \textit{Template Editor navigieren}
	\item \textbf{PLUS} Zeichen anwählen und \textit{Add Project} wählen
	\subitem Einen Namen für das Projekt angeben und speichern
	\item \textbf{PLUS} Zeichen anwählen und \textit{Add Template} wählen
	\item \textit{Name, Projekt, Device Type und Software Type} wählen und speichern
	\item Im Editor die gewünschte Konfiguration eingeben
	\item Template mittels \textit{Actions $\rightarrow$ Save} und \textit{Actions $\rightarrow$ Save} speichern
\end{enumerate}

\subsubsection{Netzwerkprofile}
\paragraph{Netzwerkprofile erstellen}
Um die erstellten Templates anwenden zu können, müssen Netzwerkprofile erstellt werden.
\begin{enumerate}
	\item Ausgehend vom DNA Center Dashboard nach \textit{Design $\rightarrow$ Network Profiles} navigieren
	\item \textit{Add Profile} klicken und \textit{Switching Profile} wählen
	\item \textit{Name} definieren
	\item Mittels \textit{Add} die gewünschten Templates hinzufügen
\end{enumerate}

\paragraph{Netzwerkprofile zuweisen}
Damit die Netzwerkprofile angewendet werden, müssen diese noch den nötigen Sites zugewiesen werden.
\begin{enumerate}
	\item Ausgehend vom DNA Center Dashboard nach \textit{Design $\rightarrow$ Network Profiles} navigieren
	\item In der Zeile des gewünschten Profils auf \textit{Sites} klicken
	\item Im neuen Popup alle nötigen Sites über das Multidropdown-Menü hinzufügen
	\item Mit \textit{Save} das Netzwerkprofil speichern
\end{enumerate}

\subsection{Policies}

\subsubsection{Virtual Network}
Virtuelle Netzwerke dienen der Isolierung der Netzwerkbenutzer und dienen somit der Sicherheit. Standardmässig können Hosts in unterschiedlichen virtuellen Netzwerken nicht miteinander kommunizieren. Mit Hilfe von virtuellen Netzwerken kann das physische Netzwerk in mehrere logische Netzwerk geteilt werden. Ein typischer Anwendungsfall ist die Segmentierung von Gästen, Mitarbeitern und Kontraktor in getrennte Gruppen, so dass der Zugriff nur auf Teile des Netzwerkes erlaubt oder eingeschränkt werden kann. Die verschiedenen Arten von Netzwerken sind:

\begin{itemize}
	\item Gast-Netzwerk: Netzwerkverbindungen, die von einem Unternehmen zur Verfügung gestellt werden, um seinen Gästen den Zugang zum Internet und zum eigenen Unternehmen zu ermöglichen, ohne die Sicherheit der Unternehmens Infrastruktur zu beeinträchtigen. Gäste können auf das Internet zugreifen, aber nicht auf interne Anwendungen.
	\item Mitarbeiter-Netzwerk: Netzwerkverbindungen, die den Zugriff auf das Internet und interne Anwendungen ermöglichen. Diese Gruppe kann weiter segmentiert werden, um Zugriffe innerhalb des Firmennetzwerks zu regeln und für spezifische Benutzer und Gruppen einzuschränken.
	\item Kontraktor-Netzwerk: Netzwerkverbindung, die es den Benutzern ermöglicht, auf das Internet und auf unternehmensspezifische Anwendungen innerhalb des Unternehmensnetzwerks zuzugreifen. 
\end{itemize}

\paragraph{Virtual Network hinzufügen}
\begin{enumerate}
	\item Ausgehend vom DNA Center Dashboard nach \textit{Policy $\rightarrow$ Virtual Network} navigieren.
	\item \textbf{PLUS} Zeichen anwählen
	\item \textit{Virtual Network Name} eingeben
	\item Scalable Groups per Drap\&Drop in das Virtual Network ziehen
	\item Mit \textit{Save} speichern
\end{enumerate}

\subsubsection{Scalable Group}
Scalable Groups umfassen eine Gruppierung von Benutzern, Endgeräten oder Ressourcen, die dieselben Anforderungen an die Zugriffskontrolle stellen. Diese Gruppen, in Cisco ISE als Sicherheitsgruppen oder SGs bekannt, werden auf dem Cisco ISE definiert. 

\paragraph{Scalable Group hinzufügen}
\begin{enumerate}
	\item Ausgehend vom DNA Center Dashboard nach \textit{Policy $\rightarrow$ Registry $\rightarrow$ Scalable Groups} navigieren
	\item \textit{Add Groups} wählen
	\item In Cisco ISE einloggen
	\item \textit{+ Add} anklicken
	\item \textit{Name, Icon und Beschreibung} eingeben 
	\item Mit \textit{Submit} speichern
\end{enumerate}
Die erstellte Scalable Group ist nun auch im DNA Center verfügbar und kann verwendet werden um Policies zu definieren.

\subsubsection{Group-based Access Control Policy}
\textit{Group-based Access Control Policies} regeln die Kommunikation zwischen Scalable Groups. Diese Policies können im DNA Center definiert werden und werden mit dem ISE synchronisiert, damit diese den Netzwerkgeräten zur Verfügung stehen.

Das folgende Beispiel zeigt den Prozess der Authentifizierung und Zugriffskontrolle, den ein Benutzer durchläuft, wenn er sich in das Netzwerk einloggt:
\begin{enumerate}
	\item Ein Benutzer verbindet sich mit dem Netzwerk.
	\item Der Benutzer authentifiziert sich am ISE.
	\item Der Switch lädt alle relevanten SGTs und SGACLs vom ISE.
	\item  Dem Benutzer wird der Zugang zu bestimmten Benutzern oder Geräten auf Grundlage der definierten Policies gewährt.
\end{enumerate}

\paragraph{Workflow}
Workflow zur Konfiguration einer gruppenbasierten Zugriffskontrollrichtlinie.

\begin{table}[H]
	\rowcolors{2}{gray!25}{white}
	\centering
	\begin{tabularx}{\textwidth}{| p{2 cm} | X |}
		\rowcolor{gray!50}
		\hline
		\textbf{Schritt} & \textbf{Aktion}  \\
		\hline	
		1 & Erstellen eines virtuellen Netzwerkes. Abhängig von der Konfiguration des Unternehmens und seinen Zugriffsanforderungen und -beschränkungen können die Gruppen in verschiedene virtuelle Netzwerke unterteilt werden, um eine weitere Segmentierung zu ermöglichen. \\
		\hline
		2 & Erstellen einer skalierbaren Gruppe. Nach der Integration von Cisco ISE werden die in ISE vorhandenen skalierbaren Gruppen in das DNA Center übertragen. Wenn eine skalierbare Gruppe nicht besteht, kann diese direkt angelegt werden. \\
		\hline
		3 & Erstellen eines Contracts. Ein Contract definiert eine Reihe von Regeln, die eine Aktion (erlauben oder verweigern), die Netzwerkgeräte basierend auf dem Datenverkehr durchführen, der bestimmten Protokollen oder Ports entspricht.  \\
		\hline
		4 & Erstellen einer \textit{Group-based Access Control Policy}. Die Policy definiert den Zugriffskontrollvertrag, der den Verkehr zwischen den skalierbaren Quell- und Zielgruppen regelt. \\
		\hline
	\end{tabularx}
	\caption{Workflow zur Erstellung der Access Control Policies}
	\label{tab:Workflow zur Erstellung der Access Control Policies}
\end{table}

\paragraph{Erstellen eines Contracts}
\begin{enumerate}
	\item Ausgehend vom DNA Center Dashboard nach \textit{Policy $\rightarrow$ Contracts $\rightarrow$ Access Contracts} navigieren
	\item \textit{Add Contract} klicken
	\item Im Dialogfenster des \textit{Contract Editor} kann ein Name und eine Beschreibung für den Contract erfasst werden
	\item \textit{Implicit Action} \textit{Deny} oder \textit{Permit} wählen
	\item \textit{Port/Protocol} wählen
	\item Mit \textit{Save} speichern
\end{enumerate}

\paragraph{Erstellen einer Group-Based Access Control Policy}
\begin{enumerate}
	\item Ausgehend vom DNA Center Dashboard nach \textit{Policy $\rightarrow$ Policy Administration $\rightarrow$ Group-Baed Access Control} navigieren
	\item \textit{Add Policy} klicken
	\item \textit{Name} und \textit{Contract} angeben
	\item \textit{Scalable Groups} für Source und Destination in die gewünschten Felder ziehen
	\item Mit \textit{Save} speichern
\end{enumerate}

\subsection{LAN Automation}
Die LAN Automation nimmt die Netzwerkgeräte mittels PnP in Betrieb und konfiguriert das Underlay Netzwerk für die Fabric und stellt somit das Routing innerhalb des Netzwerks sicher.

\subsubsection{Seed Device manuell konfigurieren}
Ein Seed Device ist nötig, damit die restlichen Geräte automatisch in Betrieb genommen werden können.
\begin{enumerate}
	\item VLAN Interface definieren, welches das \textit{Seed Device} mit dem Legacy Netzwerk verbindet und sicherstellt, dass das \textit{Seed Device} vom DNA Center aus erreichbar ist
	\item IP Adresse auf dem Loopback Interface konfigurieren (muss für das DNA Center erreichbar sein)
	\item Die im DNA Center konfigurierten Device Credentials setzen
	\item Ausgehend vom DNA Center Dashboard nach \textit{Discovery} navigieren
	\subitem \textit{Discovery Name} definieren
	\subitem \textit{Range} wählen und eine Range angeben in der sich die Loopback Adresse des Seed Devices befindet
	\subitem \textit{Preferred Management IP} auf \textit{Use Loopback} setzen
	\item Discovery mittels \textit{Start} starten
\end{enumerate}

Das Seed Device wird nach dem erfolgreichen Discovery im Inventory auftauchen und kann nun zur LAN Automation verwendet werden.

\subsubsection{LAN Automation durchführen}
\begin{enumerate}
	\item Ausgehend vom DNA Center Dashboard nach \textit{Provision $\rightarrow$ Devices $\rightarrow$ Inventory} navigieren
	\item \textit{LAN Automation} klicken
	\item \textit{Site, Seed Device, IP Pool} wählen
	\item Alle Ports wählen, über die weitere Netzwerkgeräte verbunden sind, die mittels \textit{LAN Automation} konfiguriert werden sollen
	\item \textit{LAN Automation} mittels Klick auf \textit{Start} starten
\end{enumerate}

Nun wird automatisch ein DHCP Server auf dem Seed Device konfiguriert, der den Geräten mitteilt wo sich der PnP Server befindet.

\begin{enumerate}
	\item Konfiguration eines Devices mittels \textit{write erase} löschen
	\item Gerät mittels \textit{reload} neu starten
	\subitem Das Gerät wird während dem Start den PnP Server erkennen und sich automatisch konfigurieren
	\item Die Schritte 1 und 2 wiederholen, bis alle Geräte konfiguriert sind
\end{enumerate}

Die Geräte sollten nacheinander in Betrieb genommen werden. Wird dies parallel gemacht, kann es zu Problemen mit PnP kommen.

\subsection{Provisioning}
\subsubsection{Netzwerkkomponenten Provisionieren}

\subsubsection{Fabric erstellen}
\paragraph{Border + CP festlegen}
\paragraph{Intermediate Nodes festlegen}
\paragraph{Edge Nodes festlegen}

\subsubsection{Host Onboarding}

\paragraph{Virtual Networks auswählen}
~\\
Um ein Virtual Network verwenden zu können, muss dieses in der entsprechenden Fabric zuerst aktiviert und ein IP Pool zugewiesen werden. 
\begin{enumerate}
	\item Ausgehend vom DNA Center Dashboard nach \textit{Provision $\rightarrow$ Fabric $\rightarrow$ FABRIC\_NAME $\rightarrow$ Host Onboarding} navigieren. 
	\item Im Abschnitt \textit{Virtual Networks} werden nun alle gewünschten \textit{Virtual Networks} ausgewählt. 
	\item Im neuen Popup wird der gewünschte IP Pool ausgewählt. 
	\item Der Dialog wird mit einem Klick auf \textit{Update} geschlossen.
\end{enumerate}


\paragraph{Ports konfigurieren}
~\\
Für jeden Port, der nicht bereits für die Konnektivität zwischen den Fabric Nodes verwendet wird, kann ein Addresspool, eine Gruppe und eine Authentifizierungsmethode (Siehe: \ref{authentifizierungsmethoden}) definiert werden. 
\begin{enumerate}
	\item Ausgehend vom DNA Center Dashboard nach \textit{Provision $\rightarrow$ Fabric $\rightarrow$ FABRIC\_NAME $\rightarrow$ Host Onboarding} navigieren. 
	\item Im \textit{Select Port Assignment} werden die zu konfigurierende Ports \textit{Virtual Networks} ausgewählt. 
	\item Für die ausgewählten Ports wird nun ein \textit{Address Pool}, \textit{Scalable Group}, \textit{Voice Pool} und eine Authentifizierungsmethode gewählt. 
	\item Die Änderungen werden mit einem Klick auf \textit{Save} gespeichert. 
\end{enumerate}



