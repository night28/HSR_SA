\section{Use Cases}

\subsection{Use Cases Brief}
\subsubsection{UC01: Definierung von Benutzer und Geräteprofilen}
Ein Administrator definiert die Profile für Benutzer, Gruppen oder Geräte, sodass diese auf alle nötigen Ressourcen zugreifen können, unberechtigter Zugriff aber verhindert wird.

\subsubsection{UC02: Backup and Restore DNA Center}
Auf Grund eines Problems des DNA-Centers muss die Appliance ausgetauscht oder auf einen vorherigen Konfigurationsstand zurückgesetzt werden. Um eine Neukonfiguration des Systems zu verhindern, wird eine zuvor gesicherte Konfiguration wiederhergestellt.

\subsubsection{UC03: Reporting}
Es werden regelmässig Reports über relevante Netzwerkaktivitäten erstellt und den zuständigen Personen via Mail und/oder Slack zugestellt.

\subsubsection{UC04: Hardware Ersatz}
Ein Switch muss auf Grund eines Hardwaredefekts oder ähnlichen Gründen ausgetauscht werden.

\subsubsection{UC05: Benutzermobilität}
Ein User ändert seinen Arbeitsplatz, das Gebäude oder den Arbeitsort. Er muss an allen Standorten dieselben Policies erhalten und auf dieselben Ressourcen zugreifen können.

\subsubsection{UC06: Degradation}
Bearbeitung von mögliche Degradations-Szenarios mit den entsprechenden Degradationstests.

\subsubsection{UC07: Integration von nicht Fabric Komponenten}
Netzintegration von „nicht Campus-Fabric Netzkomponenten“ (zum Beispiel traditionelle Access und Distribution Switches).

\subsubsection{UC08:  Migration von bestehendem klassichen Campus}
Migrationskonzept bestehende CampusLAN Lösung zu einem Campus-Fabric Lösung mit DNA-Center

\subsubsection{UC09: Einsatz von SGT}
Einsatz von SGT zusammen mit VXLAN (Netzdesign, Design-Rules, Transport innerhalb und aussehlab des Fabrics, Schnittstelle L2/L3 und Überführung des IP-Konnektivität an einem IP-Backbone zB MPLS VPN).

\subsubsection{UC10: Infoblox}
Integration Infoblox DDI (DNS, DHCP and IP address management) mit dem DNA-Center für die Provisionierung von IP-Adresse für das Management von neuen Netzkomponenten in die Fabric (zB Access-Switches, usw).

\subsection{Use Cases Fully Dressed}
\subsubsection{UC01: Definierung von Benutzer und Geräteprofilen}
\begin{table}[H]
	\rowcolors{2}{gray!25}{white}
	\centering
	\begin{tabularx}{\textwidth}{l | X}
		Primary Actor      & Administrator        \\
		\hline
		Beschreibung       & Ein Administrator definiert die Profile für Benutzer, Gruppen oder Geräte, sodass diese auf alle nötigen Ressourcen zugreifen können, unberechtigter Zugriff aber verhindert wird. \\ 
		\hline
		Stakeholders       &  
		\begin{itemize}	
			\item Administrator
			\item User
		\end{itemize}              \\
		\hline
		Preconditions      & 
		\begin{itemize}	
			\item DNA Center komplett konfiguriert
			\item ISE konfiguriert und mit DNA Center verbunden
		\end{itemize}  \\
		\hline
		Postconditions     & 
		\begin{itemize}	
			\item User kann auf all nötigen Ressourcen zugreifen
			\item Zugriffe auf nicht berechtige Ressourcen werden blockiert
		\end{itemize}  \\
		\hline
		Main Success Story & 
		\begin{enumerate}
			\item Profil wird definiert
			\item Profil wird Usern oder Geräten zugewiesen
			\item Entsprechende Geräte und Benutzer haben Zugriff auf benötigte Ressourcen (und keine zusätzlichen)
		\end{enumerate}
		\\
		\hline
		Alternative Flows  & 
		\begin{itemize}
			\item[1a.] Definitionen fehlen
			\begin{enumerate}
				\item Netzwerksegmente oder Ressourcen definieren
				\item Profil definieren
			\end{enumerate}
			\item[2a.] User oder Geräte fehlen
			\begin{enumerate}
				\item User oder Geräte erfassen
				\item Profil wird Usern oder Geräten zugewiesen
			\end{enumerate}
		\end{itemize}
	\end{tabularx}
	\caption{UC01 Fully Dressed}
	\label{tab:UC01}
\end{table}

\subsubsection{UC02: Backup and Restore DNA Center}
\begin{table}[H]
	\rowcolors{2}{gray!25}{white}
	\centering
	\begin{tabularx}{\textwidth}{l | X}
		Primary Actor   & Netzwerkadministrator        \\
		\hline
		Beschreibung   & Auf Grund eines Problems des DNA-Centers muss die Appliance ausgetauscht oder auf einen vorherigen Konfigurationsstand zurückgesetzt werden. Um eine Neukonfiguration des Systems zu verhindern, wird eine zuvor gesicherte Konfiguration wiederhergestellt.  \\ 
		\hline
		Stakeholders       & Alle Netzwerkbenutzer \\ 
		Preconditions      &
		\begin{itemize}	
			\item Ein Backup der DNA Center Konfiguration existiert
		\end{itemize}  \\
		\hline
		Postconditions     & 
		\begin{itemize}	
			\item Appliance läuft mit einer zuvor gesicherten Konfiguration
		\end{itemize}  \\
		\hline
		Main Success Story & 
		\begin{enumerate}
			\item Passendes Backup wählen
			\item Appliance auf den Stand des Backups zurücksetzen
		\end{enumerate}
		\\
		\hline
		Alternative Flows  & -
	\end{tabularx}
	\caption{UC02 Fully Dressed}
	\label{tab:UC02}
\end{table}

\subsubsection{UC03: Reporting}
\begin{table}[H]
	\rowcolors{2}{gray!25}{white}
	\centering
	\begin{tabularx}{\textwidth}{l | X}
		Primary Actor   & Netzwerkadministrator        \\
		\hline
		Beschreibung   & Es werden regelmässig Reports über relevante Netzwerkaktivitäten erstellt und den zuständigen Personen via Mail und/oder Slack zugestellt  \\ 
		\hline
		Stakeholders       & 
		\begin{itemize}
			\item Netzwerkadministratoren
			\item Management
		\end{itemize} \\ 
		Preconditions      &
		\begin{itemize}	
			\item Alle nötigen Daten zur Erstellung der Reports stehen im DNA Center zur Verfügung.
		\end{itemize}  \\
		\hline
		Postconditions     & 
		\begin{itemize}	
			\item Definierte Benutzer erhalten regelmässige Reports
		\end{itemize}  \\
		\hline
		Main Success Story & 
		\begin{enumerate}
			\item Relevante Informationen aus dem DNA Center werden erfasst
			\item Informationen werden aufbereitet, Report wird generiert
			\item Report wird per Mail an alle definierten Personen
		\end{enumerate}
		\\
		\hline
		Alternative Flows  & 
		\begin{itemize}
			\item[3a.] Alternativer Messenger
			\begin{enumerate}
				\item Report wird via Slack an alle definierten Personen gesendet
			\end{enumerate}
		\end{itemize}
	\end{tabularx}
	\caption{UC03 Fully Dressed}
	\label{tab:UC03}
\end{table}

\subsubsection{UC04: Hardware Ersatz}
\begin{table}[H]
	\rowcolors{2}{gray!25}{white}
	\centering
	\begin{tabularx}{\textwidth}{l | X}
		Primary Actor   & Netzwerkadministrator        \\
		\hline
		Beschreibung   & Ein Switch muss auf Grund eines Hardwaredefekts oder ähnlichen Gründen ausgetauscht werden.  \\ 
		\hline
		Stakeholders       & 
		\begin{itemize}
			\item Netzwerkadministratoren
			\item User am betroffenen Switch
		\end{itemize} \\ 
		Preconditions      &
		\begin{itemize}	
			\item Ersatzhardware verfügbar
		\end{itemize}  \\
		\hline
		Postconditions     & 
		\begin{itemize}	
			\item Ersatzhardware hat die Funktionalität des auszutauschenden Geräts vollständig übernommen
		\end{itemize}  \\
		\hline
		Main Success Story & 
		\begin{enumerate}
			\item Auszutauschendes Gerät wird entfernt
			\item Neues Gerät wird installiert
			\item Neues Gerät wird verkabelt
			\item Neues Gerät wird im DNA Center erfasst
			\item DNA Center installiert Konfiguration des alten Geräts auf das neue
			\item Neues Gerät übernimmt Funktion des alten Geräts
			\item Altes Gerät im DNA Center entfernen
		\end{enumerate}
		\\
		\hline
		Alternative Flows  & 
		\begin{itemize}
			\item[4a.] Andere Hardware
			\begin{enumerate}
				\item Ersatzhardware ist nicht identisch mit dem alten Gerät
				\item Konfiguration wird im DNA Center angepasst
			\end{enumerate} 
		\end{itemize}
	\end{tabularx}
	\caption{UC04 Fully Dressed}
	\label{tab:UC04}
\end{table}

\subsubsection{UC05: Benutzermobilität}
\begin{table}[H]
	\rowcolors{2}{gray!25}{white}
	\centering
	\begin{tabularx}{\textwidth}{l | X}
		Primary Actor   & Mobiler Benutzer        \\
		\hline
		Beschreibung   & Ein User ändert seinen Arbeitsplatz, das Gebäude oder den Arbeitsort. Er muss an allen Standorten dieselben Policies erhalten und auf dieselben Ressourcen zugreifen können.  \\ 
		\hline
		Stakeholders       & 
		\begin{itemize}
			\item User
		\end{itemize} \\ 
		Preconditions      &
		\begin{itemize}	
			\item User / Gerät erfasst und entsprechende Policies definiert
		\end{itemize}  \\
		\hline
		Postconditions     & 
		\begin{itemize}	
			\item User kann nach einem Standortwechsel alle Ressourcen verwenden, die ihm auch vor dem Wechsel zur Verfügung standen
		\end{itemize}  \\
		\hline
		Main Success Story & 
		\begin{enumerate}
			\item User trennt Verbindung am alten Standort
			\item User verbindet sich am neuen Standort
			\item User authentifiziert sich
			\item Die SDA Lösung gewährt dem User Rechte gemäss Policies
			\item User kann auf dieselben Ressourcen zugreifen wie am alten Standort
		\end{enumerate}
		\\
		\hline
		Alternative Flows  & 
		\begin{itemize}
			\item[4a.]  Während des Standortwechsels wurden die Policies angepasst
			\begin{enumerate}
				\item User erhält Rechte gemäss aktualisierten Policies
			\end{enumerate}
		\end{itemize}
	\end{tabularx}
	\caption{UC05 Fully Dressed}
	\label{tab:UC05}
\end{table}

\subsubsection{UC06: Degradation}
\begin{table}[H]
	\rowcolors{2}{gray!25}{white}
	\centering
	\begin{tabularx}{\textwidth}{l | X}
		Primary Actor   & Netzwerkadministrator       \\
		\hline
		Beschreibung   & Es soll aufgezeigt werden, wie sich das System beim Ausfall von verschiedenen Komponenten verhält, wo Single Point of Failures liegen und wie diese allenfalls eliminiert werden können.  \\ 
		\hline
		Stakeholders       & 
		\begin{itemize}
			\item Netzwerkadministrator
			\item Netzwerkbenutzer
		\end{itemize} \\ 
		Preconditions      &
		\begin{itemize}	
			\item Netzwerkinfrastruktur läuft einwandfrei
		\end{itemize}  \\
		\hline
		Postconditions     & 
		\begin{itemize}	
			\item Ausfall oder Probleme bei einer oder mehrerer Komponenten
		\end{itemize}  \\
		\hline
		Main Success Story & 
		\begin{enumerate}
			\item Netzwerk funktioniert einwandfrei
			\item Eine oder mehrere Komponenten fallen aus oder weisen sonstige Probleme auf
			\item Netzwerkfunktionalität ist durch den Ausfall nicht beeinträchtigt
			\item Fehler wird behoben, System wieder im Sollzustand
		\end{enumerate}
		\\
		\hline
		Alternative Flows  & 
		\begin{itemize}
			\item[3a.]  Durch den Ausfall kommt es zu einer Störung im Netzwerk
			\begin{enumerate}
				\item Was sind die genauen Auswirkungen? Wer ist betroffen?
				\item Wie kann die Funktionalität wiederhergestellt werden?
				\item Kann die Fehlerursache verhindert werden?
			\end{enumerate}
			\item[3b.]  Es kommt zum kompletten Ausfall des Netzwerks
			\begin{enumerate}
				\item Was sind die genauen Auswirkungen?
				\item Wie kann die Funktionalität wiederhergestellt werden?
				\item Kann die Fehlerursache verhindert werden?
			\end{enumerate}
		\end{itemize}
		\\
		\hline
		Mögliche Szenarien  & 
		\begin{itemize}
			\item Ausfall eines Edge Nodes
			\item Ausfall eines Intermediate Nodes
			\item Ausfall eines Border/Controller Nodes
			\subitem Wenn 1 Node vorhanden ist
			\subitem Wenn 2 Nodes vorhanden sind
			\item Ausfall DNA Center Appliance
			\item Ausfall ISE
			\item Ausfall Infoblox
			\item Ausfall WLC
			\item Ausfall einer physischen Netzwerkleitung
		\end{itemize}
	\end{tabularx}
	\caption{UC06 Fully Dressed}
	\label{tab:UC06}
\end{table}

\subsubsection{UC07: Integration von nicht Fabric Komponenten}
\begin{table}[H]
	\rowcolors{2}{gray!25}{white}
	\centering
	\begin{tabularx}{\textwidth}{l | X}
		Primary Actor   & Netzwerkadministrator       \\
		\hline
		Beschreibung   & Die Fabric muss mit Komponenten, die nicht der Fabric angehören kommunizieren können.  \\ 
		\hline
		Stakeholders       & 
		\begin{itemize}
			\item Netzwerkadministrator
			\item Benutzer
		\end{itemize} \\ 
		Preconditions      &
		\begin{itemize}	
			\item Fabric funktioniert
			\item Es sind Komponenten oder Teile des Netzwerks vorhanden, die nicht zu einer Fabric gehören.
		\end{itemize}  \\
		\hline
		Postconditions     & 
		\begin{itemize}	
			\item Kommunikation funktioniert auch über nicht-Fabric Komponenten hinweg
			\item Policies können auch bei Kommunikation über nicht-Fabric Komponenten angewendet werden
		\end{itemize}  \\
		\hline
		Main Success Story & 
		\begin{enumerate}
			\item Ein User kommuniziert mit Ressourcen ausserhalb der Fabric
			\item Kommunikation funktioniert einwandfrei
			\item Policies können wie bei der Kommunikation innerhalb der Fabric angewendet werden
		\end{enumerate}
		\\
		\hline
		Alternative Flows  & 
		\begin{itemize}
			\item[1a.]  Ein User ausserhalb der Fabric will mit Ressourcen innerhalb der Fabric kommunizieren
			\item[2a.] Kommunikation funktioniert einwandfrei
			\item[3a.] Policies können wie bei der Kommunikation innerhalb der Fabric angewendet werden
		\end{itemize}
	\end{tabularx}
	\caption{UC07 Fully Dressed}
	\label{tab:UC07}
\end{table}

\subsubsection{UC08: Migration von bestehenden klassichen Campus}
\begin{table}[H]
	\rowcolors{2}{gray!25}{white}
	\centering
	\begin{tabularx}{\textwidth}{l | X}
		Primary Actor   & Netzwerkadministrator       \\
		\hline
		Beschreibung   & Ein bestehendes Netzwerk nach klassischem Campusdesign soll in eine moderne Fabric migriert werden  \\ 
		\hline
		Stakeholders       & 
		\begin{itemize}
			\item Netzwerkadministrator
			\item Netzwerkbenutzer
		\end{itemize} \\ 
		Preconditions      &
		\begin{itemize}	
			\item Netzwerk nach klassischem Campusdesign existiert und funktioniert einwandfrei
			\item DNA Center Appliance inkl. aller Abhängigkeiten ist vorhanden
			\item Netzwerkkomponenten sind fähig in einer Fabric verwendet zu werden
		\end{itemize}  \\
		\hline
		Postconditions     & 
		\begin{itemize}	
			\item Fabric ist erstellt
			\item DNA Center läuft und verwaltet Fabric(s)
			\item Policies, die in der traditionellen Infrastruktur vorhanden waren funktionieren weiterhin
		\end{itemize}  \\
		\hline
		Main Success Story & 
		\begin{enumerate}
			\item Bestehende Infrastruktur wird analysiert und inventarisiert
			\item DNA Center wird aufgesetzt (inkl. aller Abhängigkeiten)
			\item User, Gruppen, Policies etc. werden in DNA Center übernommen
			\item Falls nötig wird das Netzwerkdesign angepasst
			\item Downtime wird geschätzt und organisatorische Massnahmen werden getroffen. 
			\item Bestehende Netzwerkgeräte werden in die Fabric übernommen
			\item Benutzer können Fabric analog der traditionellen Infrastruktur nutzen
		\end{enumerate}
		\\
		\hline
		Alternative Flows  & -
	\end{tabularx}
	\caption{UC08 Fully Dressed}
	\label{tab:UC08}
\end{table}

\subsubsection{UC09: Einsatz von SGT}
\begin{table}[H]
	\rowcolors{2}{gray!25}{white}
	\centering
	\begin{tabularx}{\textwidth}{l | X}
		Primary Actor   & Administrator       \\
		\hline
		Beschreibung   & Im DNA Center können über das ISE Panel definierte SGT Gruppen hinzugefügt und angepasst werden.  \\ 
		\hline
		Stakeholders       & 
		\begin{itemize}
			\item Administrator
		\end{itemize} \\ 
		Preconditions      & 
		\begin{itemize}	
			\item DNA Center muss mit dem ISE verbunden sein und alle ISE SGT-Gruppen und -Geräte müssen im DNA Center vorhanden sein
		\end{itemize}  \\
		\hline
		Postconditions     & 
		\begin{itemize}	
			\item SGT Gruppen ersichtlich
		\end{itemize}  \\
		\hline
		Main Success Story & 
		\begin{enumerate}
			\item Login auf DNA Center
			\item Unter Systemeinstellungen Cisco ISE Panel auswählen
			\item Unter Policy / Registry / Scalable Groups können neue SGT Gruppen hinzugefügt werden
		\end{enumerate}
		\\
		\hline
		Alternative Flows  & 
		\begin{itemize}
			\item[1a.]  
			\begin{enumerate}
				\item 
			\end{enumerate}
		\end{itemize}
	\end{tabularx}
	\caption{UC09 Fully Dressed}
	\label{tab:UC09}
\end{table}

\subsubsection{UC10: Infoblox}
\begin{table}[H]
	\rowcolors{2}{gray!25}{white}
	\centering
	\begin{tabularx}{\textwidth}{l | X}
		Primary Actor   & Administrator       \\
		\hline
		Beschreibung   & Infoblox ist die IP-Adressmanagement-Lösung (IPAM) für das Cisco Digital Network Architecture (DNA) Center. IP-Adresspools werden zwischen DNA Center und Infoblox synchronisiert. Mit dieser Integration kann die Zuweisung von IP-Adressen automatisiert werden, was eine richtlinienbasierte Bereitstellung in einem einzigen Vorgang ermöglicht und so die Betriebseffizienz verbessert.  \\ 
		\hline
		Stakeholders       & 
		\begin{itemize}
			\item Administrator
		\end{itemize} \\ 
		Preconditions      &
		\begin{itemize}	
			\item Infoblox Server ist eingerichtet
		\end{itemize}  \\
		\hline
		Postconditions     & 
		\begin{itemize}	
			\item Unter Design / Network Settings / IP Address Pools sind nun die IP-Adressen ersichtlich.
			\item Anpassungen an Addresspool werdne zwischen DNA Center und Infoblox synchronisiert
			\item Infoblox verwendet die im DNA Center erstellten Infos für weitere Dienste wie DNS oder DHCP
		\end{itemize}  \\
		\hline
		Main Success Story & 
		\begin{enumerate}
			\item Login auf DNA Center
			\item Unter Settings / IP Adress Manager kann ein Infoblox Server hinterlegt werden.
			\item Unter Design / Network Settings / IP Address Pools können nun die IP-Adressen angezeigt werden
		\end{enumerate}
		\\
		\hline
		Alternative Flows  & 
		\begin{itemize}
			\item[1a.] Direkt nach Installation Infoblox Server bei erstem Konfigurations-Wizard hinzufügen
			\begin{enumerate}
				\item Login auf DNA Center
				\item IP Adress Manager angeben (Server Name, Server URL, Username, Password, Provider)
			\end{enumerate}
			\item[1b.] Schritt in erstem Konfigurations-Wizard überspringen und Infoblox mit nachfolgenden Schritten hinzufügen.
		\end{itemize}
	\end{tabularx}
	\caption{UC10 Fully Dressed}
	\label{tab:UC10}
\end{table}


