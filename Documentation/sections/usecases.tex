\section{Use Cases}

\subsection{Use Cases Brief}
\subsubsection{UC01: Definierung von Benutzer und Geräteprofilen}
Ein Administrator definiert die Profile für Benutzer, Gruppen oder Geräte, sodass diese auf alle nötigen Ressourcen zugreifen können, unberechtigter Zugriff aber verhindert wird.

\subsubsection{UC02: Gastzugang}
Ein Gast (unbekannter User mit unbekanntem Gerät) steckt sich im Netzwerk an. Er erhält Zugriff auf alle definierten Ressourcen. Im einfachsten Fall einfach Internetzugriff.

\subsubsection{UC03: Backup and Restore DNA Center}
Auf Grund eines Problems des DNA-Centers muss die Appliance ausgetauscht oder auf einen vorherigen Konfigurationsstand zurückgesetzt werden. Um eine Neukonfiguration des Systems zu verhindern, wird eine zuvor gesicherte Konfiguration wiederhergestellt.

\subsubsection{UC04: Reporting}
Es werden regelmässig Reports über relevante Netzwerkaktivitäten erstellt und den zuständigen Personen via Mail und/oder Slack zugestellt.

\subsubsection{UC05: Hardware Ersatz}
Ein Switch muss auf Grund eines Hardwaredefekts oder ähnlichen Gründen ausgetauscht werden.

\subsubsection{UC06: Benutzermobilität}
Ein User ändert seinen Arbeitsplatz, das Gebäude oder den Arbeitsort. Er muss an allen Standorten dieselben Policies erhalten und auf dieselben Ressourcen zugreifen können.

\subsubsection{UC07: Degradation}
Bearbeitung von mögliche Degradations-Szenarios mit den entsprechenden Degradationstests.

\subsubsection{UC08: Integration von nicht Fabric Komponenten}
Netzintegration von „nicht Campus-Fabric Netzkomponenten“ (zum Beispiel traditionelle Access und Distribution Switches).

\subsubsection{UC09:  Migration von bestehenden klassichen Campus}
Migrationskonzept bestehende CampusLAN Lösung zu einem Campus-Fabric Lösung mit DNA-Center

\subsubsection{UC10: Einsatz von SGT}
Einsatz von SGT zusammen mit VXLAN (Netzdesign, Design-Rules, Transport innerhalb und aussehlab des Fabrics, Schnittstelle L2/L3 und Überführung des IP-Konnektivität an einem IP-Backbone zB MPLS VPN).

\subsubsection{UC11: Infoblox}
Integration Infoblox DDI mit dem DNA-Center für die Provisionierung von IP-Adresse für das Management von neuen Netzkomponenten in die Fabric (zB Access-Switches, usw).

\subsection{Use Cases Fully dressed}
\subsubsection{UC01: Definierung von Benutzer und Geräteprofilen}
\begin{table}[H]
	\rowcolors{2}{gray!25}{white}
	\centering
	\begin{tabularx}{\textwidth}{l | X}
		Primary Actor      & Administrator        \\
		\hline
		Beschreibung       & Ein Administrator definiert die Profile für Benutzer, Gruppen oder Geräte, sodass diese auf alle nötigen Ressourcen zugreifen können, unberechtigter Zugriff aber verhindert wird. \\ 
		\hline
		Stakeholders       &  
		\begin{itemize}	
			\item Administrator
			\item User
		\end{itemize}              \\
		\hline
		Preconditions      & 
		\begin{itemize}	
			\item DNA Center komplett konfiguriert
			\item ISE konfiguriert und mit DNA Center verbunden
		\end{itemize}  \\
		\hline
		Postconditions     & 
		\begin{itemize}	
			\item User kann auf all nötigen Ressourcen zugreifen
			\item Zugriffe auf nicht berechtige Ressourcen werden blockiert
		\end{itemize}  \\
		\hline
		Main Success Story & 
		\begin{enumerate}
			\item Profil wird definiert
			\item Profil wird Usern oder Geräten zugewiesen
			\item Entsprechende Geräte und Benutzer haben Zugriff auf benötigte Ressourcen (und keine zusätzlichen)
		\end{enumerate}
		\\
		\hline
		Alternative Flows  & 
		\begin{itemize}
			\item[1a.] Definitionen fehlen
			\begin{enumerate}
				\item Netzwerksegmente oder Ressourcen definieren
				\item Profil definieren
			\end{enumerate}
			\item[2a.] User oder Geräte fehlen
			\begin{enumerate}
				\item User oder Geräte erfassen
				\item Profil wird Usern oder Geräten zugewiesen
			\end{enumerate}
		\end{itemize}
	\end{tabularx}
	\caption{UC01 Fully Dressed}
	\label{tab:UC01}
\end{table}
\subsubsection{UC02: Gastzugang}
\begin{table}[H]
	\rowcolors{2}{gray!25}{white}
	\centering
	\begin{tabularx}{\textwidth}{l | X}
		Primary Actor   & Guest        \\
		\hline
		Beschreibung   & Ein Gast (unbekannter User mit unbekanntem Gerät) steckt sich im Netzwerk an. Er erhält Zugriff auf alle definierten Ressourcen. Im einfachsten Fall einfach Internetzugriff.  \\ 
		\hline
		Stakeholders       & - \\ 
		Preconditions      &
		\begin{itemize}	
			\item Profil für Gastzugriff definiert
			\item Gast ist mit dem Netzwerk verbunden
		\end{itemize}  \\
		\hline
		Postconditions     & 
		\begin{itemize}	
			\item Gast hat Zugriff auf definierte Ressourcen
			\item Gast hat keinen Zugriff auf interne Ressourcen
		\end{itemize}  \\
		\hline
		Main Success Story & 
		\begin{enumerate}
			\item Gast verbindet sich mit dem Netzwerk
			\item Gast erhält Zugriff auf defnierte Ressourcen
			\item Gast verlässt das Netzwerk
		\end{enumerate}
		\\
		\hline
		Alternative Flows  & -
	\end{tabularx}
	\caption{UC02 Fully Dressed}
	\label{tab:UC02}
\end{table}

\subsubsection{UC03: Backup and Restore DNA Center}
\begin{table}[H]
	\rowcolors{2}{gray!25}{white}
	\centering
	\begin{tabularx}{\textwidth}{l | X}
		Primary Actor   & Netzwerkadministrator        \\
		\hline
		Beschreibung   & Auf Grund eines Problems des DNA-Centers muss die Appliance ausgetauscht oder auf einen vorherigen Konfigurationsstand zurückgesetzt werden. Um eine Neukonfiguration des Systems zu verhindern, wird eine zuvor gesicherte Konfiguration wiederhergestellt.  \\ 
		\hline
		Stakeholders       & Alle Netzwerkbenutzer \\ 
		Preconditions      &
		\begin{itemize}	
			\item Ein Backup der DNA Center Konfiguration existiert
		\end{itemize}  \\
		\hline
		Postconditions     & 
		\begin{itemize}	
			\item Appliance läuft mit einer zuvor gesicherten Konfiguration
		\end{itemize}  \\
		\hline
		Main Success Story & 
		\begin{enumerate}
			\item Passendes Backup wählen
			\item Appliance auf den Stand des Backups zurücksetzen
		\end{enumerate}
		\\
		\hline
		Alternative Flows  & -
	\end{tabularx}
	\caption{UC03 Fully Dressed}
	\label{tab:UC03}
\end{table}

\subsubsection{UC04: Reporting}
\begin{table}[H]
	\rowcolors{2}{gray!25}{white}
	\centering
	\begin{tabularx}{\textwidth}{l | X}
		Primary Actor   & Netzwerkadministrator        \\
		\hline
		Beschreibung   & Es werden regelmässig Reports über relevante Netzwerkaktivitäten erstellt und den zuständigen Personen via Mail und/oder Slack zugestellt  \\ 
		\hline
		Stakeholders       & 
		\begin{itemize}
			\item Netzwerkadministratoren
			\item Management
		\end{itemize} \\ 
		Preconditions      &
		\begin{itemize}	
			\item Alle nötigen Daten zur Erstellung der Reports stehen im DNA Center zur Verfügung.
		\end{itemize}  \\
		\hline
		Postconditions     & 
		\begin{itemize}	
			\item Definierte Benutzer erhalten regelmässige Reports
		\end{itemize}  \\
		\hline
		Main Success Story & 
		\begin{enumerate}
			\item Relevante Informationen aus dem DNA Center werden erfasst
			\item Informationen werden aufbereitet, Report wird generiert
			\item Report wird per Mail an alle definierten Personen
		\end{enumerate}
		\\
		\hline
		Alternative Flows  & 
		\begin{itemize}
			\item[3a.] Alternativer Messenger
			\begin{enumerate}
				\item Report wird via Slack an alle definierten Personen gesendet
			\end{enumerate}
		\end{itemize}
	\end{tabularx}
	\caption{UC04 Fully Dressed}
	\label{tab:UC04}
\end{table}

\subsubsection{UC05: Hardware Ersatz}
\begin{table}[H]
	\rowcolors{2}{gray!25}{white}
	\centering
	\begin{tabularx}{\textwidth}{l | X}
		Primary Actor   & Netzwerkadministrator        \\
		\hline
		Beschreibung   & Ein Switch muss auf Grund eines Hardwaredefekts oder ähnlichen Gründen ausgetauscht werden.  \\ 
		\hline
		Stakeholders       & 
		\begin{itemize}
			\item Netzwerkadministratoren
			\item User am betroffenen Switch
		\end{itemize} \\ 
		Preconditions      &
		\begin{itemize}	
			\item Ersatzhardware verfügbar
		\end{itemize}  \\
		\hline
		Postconditions     & 
		\begin{itemize}	
			\item Ersatzhardware hat die Funktionalität des auszutauschenden Geräts vollständig übernommen
		\end{itemize}  \\
		\hline
		Main Success Story & 
		\begin{enumerate}
			\item Auszutauschendes Gerät wird entfernt
			\item Neues Gerät wird installiert
			\item Neues Gerät wird verkabelt
			\item Neues Gerät wird im DNA Center erfasst
			\item DNA Center installiert Konfiguration des alten Geräts auf das neue
			\item Neues Gerät übernimmt Funktion des alten Geräts
			\item Altes Gerät im DNA Center entfernen
		\end{enumerate}
		\\
		\hline
		Alternative Flows  & 
		\begin{itemize}
			\item[4a.] Andere Hardware
			\begin{enumerate}
				\item Ersatzhardware ist nicht identisch mit dem alten Gerät
				\item Konfiguration wird im DNA Center angepasst
			\end{enumerate} 
		\end{itemize}
	\end{tabularx}
	\caption{UC05 Fully Dressed}
	\label{tab:UC05}
\end{table}

\subsubsection{UC06: Benutzermobilität}
\begin{table}[H]
	\rowcolors{2}{gray!25}{white}
	\centering
	\begin{tabularx}{\textwidth}{l | X}
		Primary Actor   & Mobiler Benutzer        \\
		\hline
		Beschreibung   & Ein User ändert seinen Arbeitsplatz, das Gebäude oder den Arbeitsort. Er muss an allen Standorten dieselben Policies erhalten und auf dieselben Ressourcen zugreifen können.  \\ 
		\hline
		Stakeholders       & 
		\begin{itemize}
			\item User
		\end{itemize} \\ 
		Preconditions      &
		\begin{itemize}	
			\item User / Gerät erfasst und entsprechende Policies definiert
		\end{itemize}  \\
		\hline
		Postconditions     & 
		\begin{itemize}	
			\item User kann nach einem Standortwechsel alle Ressourcen verwenden, die ihm auch vor dem Wechsel zur Verfügung standen
		\end{itemize}  \\
		\hline
		Main Success Story & 
		\begin{enumerate}
			\item User trennt Verbindung am alten Standort
			\item User verbindet sich am neuen Standort
			\item User authentifiziert sich
			\item Die SDA Lösung gewährt dem User Rechte gemäss Policies
			\item User kann auf dieselben Ressourcen zugreifen wie am alten Standort
		\end{enumerate}
		\\
		\hline
		Alternative Flows  & 
		\begin{itemize}
			\item[4a.]  Während des Standortwechsels wurden die Policies angepasst
			\begin{enumerate}
				\item User erhält Rechte gemäss aktualisierten Policies
			\end{enumerate}
		\end{itemize}
	\end{tabularx}
	\caption{UC06 Fully Dressed}
	\label{tab:UC06}
\end{table}

\subsubsection{UC07: Degradation}
\begin{table}[H]
	\rowcolors{2}{gray!25}{white}
	\centering
	\begin{tabularx}{\textwidth}{l | X}
		Primary Actor   & TBA       \\
		\hline
		Beschreibung   & TBA  \\ 
		\hline
		Stakeholders       & 
		\begin{itemize}
			\item TBA
		\end{itemize} \\ 
		Preconditions      &
		\begin{itemize}	
			\item TBA
		\end{itemize}  \\
		\hline
		Postconditions     & 
		\begin{itemize}	
			\item TBA
		\end{itemize}  \\
		\hline
		Main Success Story & 
		\begin{enumerate}
			\item TBA
			\item TBA
		\end{enumerate}
		\\
		\hline
		Alternative Flows  & 
		\begin{itemize}
			\item[1a.]  TBA
			\begin{enumerate}
				\item TBA
			\end{enumerate}
		\end{itemize}
	\end{tabularx}
	\caption{UC07 Fully Dressed}
	\label{tab:UC07}
\end{table}

\subsubsection{UC08: Integration von nicht Fabric Komponenten}
\begin{table}[H]
	\rowcolors{2}{gray!25}{white}
	\centering
	\begin{tabularx}{\textwidth}{l | X}
		Primary Actor   & TBA       \\
		\hline
		Beschreibung   & TBA  \\ 
		\hline
		Stakeholders       & 
		\begin{itemize}
			\item TBA
		\end{itemize} \\ 
		Preconditions      &
		\begin{itemize}	
			\item TBA
		\end{itemize}  \\
		\hline
		Postconditions     & 
		\begin{itemize}	
			\item TBA
		\end{itemize}  \\
		\hline
		Main Success Story & 
		\begin{enumerate}
			\item TBA
			\item TBA
		\end{enumerate}
		\\
		\hline
		Alternative Flows  & 
		\begin{itemize}
			\item[1a.]  TBA
			\begin{enumerate}
				\item TBA
			\end{enumerate}
		\end{itemize}
	\end{tabularx}
	\caption{UC08 Fully Dressed}
	\label{tab:UC08}
\end{table}

\subsubsection{UC09: Migration von bestehenden klassichen Campus}
\begin{table}[H]
	\rowcolors{2}{gray!25}{white}
	\centering
	\begin{tabularx}{\textwidth}{l | X}
		Primary Actor   & TBA       \\
		\hline
		Beschreibung   & TBA  \\ 
		\hline
		Stakeholders       & 
		\begin{itemize}
			\item TBA
		\end{itemize} \\ 
		Preconditions      &
		\begin{itemize}	
			\item TBA
		\end{itemize}  \\
		\hline
		Postconditions     & 
		\begin{itemize}	
			\item TBA
		\end{itemize}  \\
		\hline
		Main Success Story & 
		\begin{enumerate}
			\item TBA
			\item TBA
		\end{enumerate}
		\\
		\hline
		Alternative Flows  & 
		\begin{itemize}
			\item[1a.]  TBA
			\begin{enumerate}
				\item TBA
			\end{enumerate}
		\end{itemize}
	\end{tabularx}
	\caption{UC09 Fully Dressed}
	\label{tab:UC09}
\end{table}

\subsubsection{UC10: Einsatz von SGT}
\begin{table}[H]
	\rowcolors{2}{gray!25}{white}
	\centering
	\begin{tabularx}{\textwidth}{l | X}
		Primary Actor   & Administrator       \\
		\hline
		Beschreibung   & Im DNA Center können über das ISE Panel definierte SGT Gruppen hinzugefügt und angepasst werden.  \\ 
		\hline
		Stakeholders       & 
		\begin{itemize}
			\item TBA
		\end{itemize} \\ 
		Preconditions      & DNA Center muss mit dem ISE verbunden sein und alle ISE SGT-Gruppen und -Geräte müssen im DNA Center vorhanden sein
		\begin{itemize}	
			\item TBA
		\end{itemize}  \\
		\hline
		Postconditions     & 
		\begin{itemize}	
			\item TBA
		\end{itemize}  \\
		\hline
		Main Success Story & 
		\begin{enumerate}
			\item Login auf DNA Center
			\item Unter Systemeinstellungen Cisco ISE Panel auswählen
			\item Unter Policy / Registry / Scalable Groups können neue SGT Gruppen hinzugefügt werden
		\end{enumerate}
		\\
		\hline
		Alternative Flows  & 
		\begin{itemize}
			\item[1a.]  TBA
			\begin{enumerate}
				\item TBA
			\end{enumerate}
		\end{itemize}
	\end{tabularx}
	\caption{UC10 Fully Dressed}
	\label{tab:UC10}
\end{table}

\subsubsection{UC11: Infoblox}
\begin{table}[H]
	\rowcolors{2}{gray!25}{white}
	\centering
	\begin{tabularx}{\textwidth}{l | X}
		Primary Actor   & TBA       \\
		\hline
		Beschreibung   & TBA  \\ 
		\hline
		Stakeholders       & 
		\begin{itemize}
			\item TBA
		\end{itemize} \\ 
		Preconditions      &
		\begin{itemize}	
			\item TBA
		\end{itemize}  \\
		\hline
		Postconditions     & 
		\begin{itemize}	
			\item TBA
		\end{itemize}  \\
		\hline
		Main Success Story & 
		\begin{enumerate}
			\item TBA
			\item TBA
		\end{enumerate}
		\\
		\hline
		Alternative Flows  & 
		\begin{itemize}
			\item[1a.]  TBA
			\begin{enumerate}
				\item TBA
			\end{enumerate}
		\end{itemize}
	\end{tabularx}
	\caption{UC11 Fully Dressed}
	\label{tab:UC11}
\end{table}

