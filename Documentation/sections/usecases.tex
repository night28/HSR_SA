\section{Use Cases}

\subsection{Use Cases Brief}
\subsubsection{UC01: }


\subsubsection{UC02: }

\subsection{Use Cases Fully dressed}
\subsubsection{UC01: Definierung von Benutzer und Geräteprofilen }
\begin{table}[H]
	\rowcolors{2}{gray!25}{white}
	\centering
	\begin{tabularx}{\textwidth}{l | X}
		Primary Actor      & Administrator        \\
		\hline
		Beschreibung       & Ein Administrator definiert die Profile für Benutzer, Gruppen oder Geräte, sodass diese auf alle nötigen Ressourcen zugreifen können, unberechtigter Zugriff aber verhindert wird. \\ 
		\hline
		Stakeholders       &  
		\begin{itemize}	
			\item Administrator
			\item User
		\end{itemize}              \\
		\hline
		Preconditions      & 
		\begin{itemize}	
			\item DNA Center komplett konfiguriert
			\item ISE konfiguriert und mit DNA Center verbunden
		\end{itemize}  \\
		\hline
		Postconditions     & 
		\begin{itemize}	
			\item User kann auf all nötigen Ressourcen zugreifen
			\item Zugriffe auf nicht berechtige Ressourcen werden blockiert
		\end{itemize}  \\
		\hline
		Main Success Story & 
		1.  Profil wird definiert\newline
		2.  Profil wird Usern oder Geräten zugewiesen\newline
		3.  Entsprechende Geräte und Benutzer haben Zugriff auf benötigte Ressourcen (und keine zusätzlichen)\newline
		\\
		\hline
		Alternative Flows  & 
		1a Definitionen fehlen \newline
		\noindent\hspace*{6mm} 1. Netzwerksegmente oder Ressourcen definieren \newline
		\noindent\hspace*{6mm} 2. Profil definieren
		\newline
		2a User oder Geräte fehlen \newline
		\noindent\hspace*{6mm} 1.User oder Geräte erfassen \newline
		\noindent\hspace*{6mm} 2. Profil wird Usern oder Geräten zugewiesen
	\end{tabularx}
	\caption{Fully Dressed UC01}
	\label{tab:UC02}
\end{table}
\subsubsection{UC02: Gastzugang}
\begin{table}[H]
	\rowcolors{2}{gray!25}{white}
	\centering
	\begin{tabularx}{\textwidth}{l | X}
		Primary Actor   & Guest        \\
		\hline
		Beschreibung   & Ein Gast (unbekannter User mit unbekanntem Gerät) steckt sich im Netzwerk an. Er erhält Zugriff auf alle definierten Ressourcen. Im einfachsten Fall einfach Internetzugriff.  \\ 
		\hline
		Stakeholders       & - \\ 
		Preconditions      &
		\begin{itemize}	
			\item Profil für Gastzugriff definiert
			\item Gast ist mit dem Netzwerk verbunden
		\end{itemize}  \\
		\hline
		Postconditions     & 
		\begin{itemize}	
			\item Gast hat Zugriff auf definierte Ressourcen
			\item Gast hat keinen Zugriff auf interne Ressourcen
		\end{itemize}  \\
		\hline
		Main Success Story & 
		1.  Gast verbindet sich mit dem Netzwerk\newline
		2.  Gast erhält Zugriff auf defnierte Ressourcen\newline
		3.  Gast verlässt das Netzwerk\newline
		\\
		\hline
		Alternative Flows  & -
	\end{tabularx}
	\caption{Fully Dressed UC02}
	\label{tab:UC02}
\end{table}

\subsubsection{UC03: Backup and Restore DNA Center}
\begin{table}[H]
	\rowcolors{2}{gray!25}{white}
	\centering
	\begin{tabularx}{\textwidth}{l | X}
		Primary Actor   & Netzwerkadministrator        \\
		\hline
		Beschreibung   & Auf Grund eines Problems des DNA-Centers muss die Appliance ausgetauscht oder auf einen vorherigen Konfigurationsstand zurückgesetzt werden. Um eine Neukonfiguration des Systems zu verhindern, wird eine zuvor gesicherte Konfiguration wiederhergestellt.  \\ 
		\hline
		Stakeholders       & Alle Netzwerkbenutzer \\ 
		Preconditions      &
		\begin{itemize}	
			\item Ein Backup der DNA Center Konfiguration existiert
		\end{itemize}  \\
		\hline
		Postconditions     & 
		\begin{itemize}	
			\item Appliance läuft mit einer zuvor gesicherten Konfiguration
		\end{itemize}  \\
		\hline
		Main Success Story & 
		1.  Passendes Backup wählen\newline
		2.  Appliance auf den Stand des Backups zurücksetzen\newline
		\\
		\hline
		Alternative Flows  & -
	\end{tabularx}
	\caption{Fully Dressed UC03}
	\label{tab:UC03}
\end{table}

\subsubsection{UC04: Reporting}
\begin{table}[H]
	\rowcolors{2}{gray!25}{white}
	\centering
	\begin{tabularx}{\textwidth}{l | X}
		Primary Actor   & Netzwerkadministrator        \\
		\hline
		Beschreibung   & Es werden regelmässig Reports über relevante Netzwerkaktivitäten erstellt und den zuständigen Personen via Mail und/oder Slack zugestellt  \\ 
		\hline
		Stakeholders       & 
		\begin{itemize}
			\item Netzwerkadministratoren
			\item Management
		\end{itemize} \\ 
		Preconditions      &
		\begin{itemize}	
			\item Alle nötigen Daten zur Erstellung der Reports stehen im DNA Center zur Verfügung.
		\end{itemize}  \\
		\hline
		Postconditions     & 
		\begin{itemize}	
			\item Definierte Benutzer erhalten regelmässige Reports
		\end{itemize}  \\
		\hline
		Main Success Story & 
		1.  Relevante Informationen aus dem DNA Center werden erfasst\newline
		2.  Informationen werden aufbereitet, Report wird generiert\newline
		3.  Report wird per Mail an alle definierten Personen \newline
		\\
		\hline
		Alternative Flows  & 
		3a Alternativer Messenger \newline
		\noindent\hspace*{6mm} 1. Report wird via Slack an alle definierten Personen gesendet \newline
	\end{tabularx}
	\caption{Fully Dressed UC04}
	\label{tab:UC04}
\end{table}

\subsubsection{UC05: Hardware Ersatz}
\begin{table}[H]
	\rowcolors{2}{gray!25}{white}
	\centering
	\begin{tabularx}{\textwidth}{l | X}
		Primary Actor   & Netzwerkadministrator        \\
		\hline
		Beschreibung   & Ein Switch muss auf Grund eines Hardwaredefekts oder ähnlichen Gründen ausgetauscht werden.  \\ 
		\hline
		Stakeholders       & 
		\begin{itemize}
			\item Netzwerkadministratoren
			\item User am betroffenen Switch
		\end{itemize} \\ 
		Preconditions      &
		\begin{itemize}	
			\item Ersatzhardware verfügbar
		\end{itemize}  \\
		\hline
		Postconditions     & 
		\begin{itemize}	
			\item Ersatzhardware hat die Funktionalität des auszutauschenden Geräts vollständig übernommen
		\end{itemize}  \\
		\hline
		Main Success Story & 
		1.  Auszutauschendes Gerät wird entfernt\newline
		2.  Neues Gerät wird installiert\newline
		3.  Neues Gerät wird verkabelt\newline
		4.  Neues Gerät wird im DNA Center erfasst\newline
		5.  DNA Center installiert Konfiguration des alten Geräts auf das neue\newline
		6.  Neues Gerät übernimmt Funktion des alten Geräts\newline
		7.  Altes Gerät im DNA Center entfernen\newline
		\\
		\hline
		Alternative Flows  & 
		4a Andere Hardware \newline
		\noindent\hspace*{6mm} 1. Ersatzhardware ist nicht identisch mit dem alten Gerät \newline
		\noindent\hspace*{6mm} 2. Konfiguration wird im DNA Center angepasst \newline
	\end{tabularx}
	\caption{Fully Dressed UC05}
	\label{tab:UC05}
\end{table}

\subsubsection{UC06: Benutzermobilität}
\begin{table}[H]
	\rowcolors{2}{gray!25}{white}
	\centering
	\begin{tabularx}{\textwidth}{l | X}
		Primary Actor   & Mobiler Benutzer        \\
		\hline
		Beschreibung   & Ein User ändert seinen Arbeitsplatz, das Gebäude oder den Arbeitsort. Er muss an allen Standorten dieselben Policies erhalten und auf dieselben Ressourcen zugreifen können.  \\ 
		\hline
		Stakeholders       & 
		\begin{itemize}
			\item User
		\end{itemize} \\ 
		Preconditions      &
		\begin{itemize}	
			\item User / Gerät erfasst und entsprechende Policies definiert
		\end{itemize}  \\
		\hline
		Postconditions     & 
		\begin{itemize}	
			\item User kann nach einem Standortwechsel alle Ressourcen verwenden, die ihm auch vor dem Wechsel zur Verfügung standen
		\end{itemize}  \\
		\hline
		Main Success Story & 
		1.  User trennt Verbindung am alten Standort\newline
		2.  User verbindet sich am neuen Standort\newline
		3.  User authentifiziert sich\newline
		4.  Die SDA Lösung gewährt dem User Rechte gemäss Policies\newline
		5.  User kann auf dieselben Ressourcen zugreifen wie am alten Standort\newline
		\\
		\hline
		Alternative Flows  & 
		4a Während des Standortwechsels wurden die Policies angepasst \newline
		\noindent\hspace*{6mm} 1.User erhält Rechte gemäss aktualisierten Policies \newline
	\end{tabularx}
	\caption{Fully Dressed UC06}
	\label{tab:UC06}
\end{table}

\subsubsection{UC07: Degradation}
\begin{table}[H]
	\rowcolors{2}{gray!25}{white}
	\centering
	\begin{tabularx}{\textwidth}{l | X}
		Primary Actor   & TBA        \\
		\hline
		Beschreibung   & TBA  \\ 
		\hline
		Stakeholders       & 
		\begin{itemize}
			\item TBA
		\end{itemize} \\ 
		Preconditions      &
		\begin{itemize}	
			\item TBA
		\end{itemize}  \\
		\hline
		Postconditions     & 
		\begin{itemize}	
			\item TBA
		\end{itemize}  \\
		\hline
		Main Success Story & TBA
		\newline
		\\
		\hline
		Alternative Flows  & 
		TBA \newline
		\newline
	\end{tabularx}
	\caption{Fully Dressed UC07}
	\label{tab:UC07}
\end{table}

\subsubsection{UC08: Integration von nicht Fabric Komponenten}
\begin{table}[H]
	\rowcolors{2}{gray!25}{white}
	\centering
	\begin{tabularx}{\textwidth}{l | X}
		Primary Actor   & TBA        \\
		\hline
		Beschreibung   & TBA  \\ 
		\hline
		Stakeholders       & 
		\begin{itemize}
			\item TBA
		\end{itemize} \\ 
		Preconditions      &
		\begin{itemize}	
			\item TBA
		\end{itemize}  \\
		\hline
		Postconditions     & 
		\begin{itemize}	
			\item TBA
		\end{itemize}  \\
		\hline
		Main Success Story & TBA
		\newline
		\\
		\hline
		Alternative Flows  & 
		TBA \newline
		\newline
	\end{tabularx}
	\caption{Fully Dressed UC08}
	\label{tab:UC08}
\end{table}

\subsubsection{UC09: Migration von bestehenden klassichen Campus}
\begin{table}[H]
	\rowcolors{2}{gray!25}{white}
	\centering
	\begin{tabularx}{\textwidth}{l | X}
		Primary Actor   & TBA        \\
		\hline
		Beschreibung   & TBA  \\ 
		\hline
		Stakeholders       & 
		\begin{itemize}
			\item TBA
		\end{itemize} \\ 
		Preconditions      &
		\begin{itemize}	
			\item TBA
		\end{itemize}  \\
		\hline
		Postconditions     & 
		\begin{itemize}	
			\item TBA
		\end{itemize}  \\
		\hline
		Main Success Story & TBA
		\newline
		\\
		\hline
		Alternative Flows  & 
		TBA \newline
		\newline
	\end{tabularx}
	\caption{Fully Dressed UC09}
	\label{tab:UC09}
\end{table}

\subsubsection{UC10: Einsatz von SGT}
\begin{table}[H]
	\rowcolors{2}{gray!25}{white}
	\centering
	\begin{tabularx}{\textwidth}{l | X}
		Primary Actor   & TBA        \\
		\hline
		Beschreibung   & TBA  \\ 
		\hline
		Stakeholders       & 
		\begin{itemize}
			\item TBA
		\end{itemize} \\ 
		Preconditions      &
		\begin{itemize}	
			\item TBA
		\end{itemize}  \\
		\hline
		Postconditions     & 
		\begin{itemize}	
			\item TBA
		\end{itemize}  \\
		\hline
		Main Success Story & TBA
		\newline
		\\
		\hline
		Alternative Flows  & 
		TBA \newline
		\newline
	\end{tabularx}
	\caption{Fully Dressed UC10}
	\label{tab:UC10}
\end{table}

\subsubsection{UC11: Netblox}
\begin{table}[H]
	\rowcolors{2}{gray!25}{white}
	\centering
	\begin{tabularx}{\textwidth}{l | X}
		Primary Actor   & TBA        \\
		\hline
		Beschreibung   & TBA  \\ 
		\hline
		Stakeholders       & 
		\begin{itemize}
			\item TBA
		\end{itemize} \\ 
		Preconditions      &
		\begin{itemize}	
			\item TBA
		\end{itemize}  \\
		\hline
		Postconditions     & 
		\begin{itemize}	
			\item TBA
		\end{itemize}  \\
		\hline
		Main Success Story & TBA
		\newline
		\\
		\hline
		Alternative Flows  & 
		TBA \newline
		\newline
	\end{tabularx}
	\caption{Fully Dressed UC11}
	\label{tab:UC11}
\end{table}
