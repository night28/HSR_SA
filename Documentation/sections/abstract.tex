\section{Abstract}
Der Abstract richtet sich an den Spezialisten auf dem entsprechenden Gebiet und 
beschreibt daher in erster Linie die (neuen, eigenen) Ergebnisse und Resultate der 
Arbeit. Es umfasst nie mehr als eine Seite, typisch sogar nur etwa 200 Worte (etwa 
20 Zeilen). Es sind keine Bilder zu verwenden.
\subsection{Ausgangslage}

\subsection{Ziele der Arbeit}
Das grundsätzliche Ziel ist eine Evaluation des Cisco Software Defined Networking. Im ersten Schritt beinhaltet das die Inbetriebnahme und Konfiguration von den folgenden Komponenten:
\begin{itemize}
	\item Cisco DNA Center Appliance
	\item Integration Infoblox (One Platform Solution für DNS, DHCP, IPAM)
	\item Integration Cisco ISE (Access and Authentification Control)
	\item Integration Campus Labor Netzwerk
\end{itemize}
Im zweiten Schritt sollen UseCases durchgespielt werden, die den folgenden Umfang abdecken:
\begin{itemize}
	\item Definierung von Benutzer- und Geräteprofile, um basierend auf Geschäftsanforderungen die Zugriffsrechte und Netzwerksegmentierung zu verwalten und so das Netzwerk sicher zu halten.
	\item DNA Analytics and Assurance für eine proaktive Überwachung, Fehlerbehebung und Optimierung des Netzwerks.
	\item IP Address Management Tool im DNA Center.
	\item Wöchentliche Reports über Campus Netzwerk-Status via E-Mail oder Slack
\end{itemize}

\subsection{Ergebnisse}
