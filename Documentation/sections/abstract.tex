\section{Abstract}
Der Abstract richtet sich an den Spezialisten auf dem entsprechenden Gebiet und 
beschreibt daher in erster Linie die (neuen, eigenen) Ergebnisse und Resultate der 
Arbeit. Es umfasst nie mehr als eine Seite, typisch sogar nur etwa 200 Worte (etwa 
20 Zeilen). Es sind keine Bilder zu verwenden.
\subsection{Ausgangslage}

\subsection{Ziele der Arbeit}
Das grundsätzliche Ziel ist eine Evaluation des Cisco Software Defined Networking. Im ersten Schritt beinhaltet das die Inbetriebnahme und Konfiguration von den folgenden Komponenten:
\begin{itemize}
	\item Cisco DNA Center Appliance
	\item Integration Infoblox (One Platform Solution für DNS, DHCP, IPAM)
	\item Integration Cisco ISE (Access and Authentification Control)
	\item Integration Campus Labor Netzwerk
\end{itemize}
Im zweiten Schritt sollen UseCases durchgespielt werden, die den folgenden Umfang abdecken:
\begin{itemize}
	\item Definierung von Benutzer- und Geräteprofile, um basierend auf Geschäftsanforderungen die Zugriffsrechte und Netzwerksegmentierung zu verwalten und so das Netzwerk sicher zu halten.
	\item DNA Analytics and Assurance für eine proaktive Überwachung, Fehlerbehebung und Optimierung des Netzwerks.
	\item IP Address Management Tool im DNA Center.
	\item Wöchentliche Reports über Campus Netzwerk-Status via E-Mail oder Slack
\end{itemize}

\subsection{Ergebnisse}




**** \\
Ziel dieser Studienarbeit war die Evaluation der Software Defined Access Lösung von Cisco für die Führungsunterstützungsbasis der Schweizer Armee. Der erste Schwerpunkt lag dabei auf der Installation des Cisco Digital Network Architecture (DNA) Centers und der Konfiguration sowie Integration eines Campus Netzwerkes. Das Campus Netzwerk verwendet für das IPAM Cisco Infoblox und für die Benutzerverwaltung sowie Zugriffskontrolle Cisco Identity Services Engine (ISE).

Der zweite Schwerpunkt lag auf der Definierung von Benutzer- und Geräteprofilen, um den einzelnen Mitarbeitern der Führungsunterstützungsbasis der Schweizer Armee den Netzwerkzugriff sicherzustellen und die Zugriffsrechte der einzelnen Mitarbeiter oder eines Teams zu regeln. Zusätzlich hierzu sollte mit DNA Analytics und Assurance eine proaktive Überwachung, Fehlerbehebung und Optimierung des Netzwerkes sichergestellt werden. Mit diesen Informationen sollten wöchentliche Reports über den Campus Netzwerk Status in einem E-Mail oder in einer Slack Message versendet werden.
\\

Die Implementation des Campus Netzwerkes in das DNA Center erwies sich als schwieriger als gedacht. Die LAN Automation des Underlays konnte erst nach mehreren Anläufen durchgeführt werden. So wurde der ganze Provisionierungsprozesse verzögert, konnte aber anschliessend ausgeführt werden. Die Definierung der Benutzer- und Geräteprofile wurde an die Klassifizierungsstufen der Schweizer Armee angelehnt.
\\

Abschliessend kann gesagt werden, das für die Installation und Konfiguration des DNA Centers ein Green Field vorliegen muss. Es kann keine bestehendes Netzwerk migriert werden. Darüber hinaus müssen mehrere Tage wenn nicht Wochen für diese Umsetzung eingerechnet werden, da die kompatiblen Versionen genaustens eingehalten werden müssen und das Netzwerk genau nach Anleitung aufgesetzt werden muss.


