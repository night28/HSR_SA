\section{Ergebnisdiskussion}



\subsection{Zielsetzungen}
Wie in der Architektur ersichtlich, wurden zwei Standorte geplant. Aufgrund der aufgetretenen Probleme und Bugs, wurde entschieden sich nur auf die Seite Rapperswil mit mehreren Devices und zwei Gebäuden zu konzentrieren. Hier konnte eine laufende Fabric erstellt und Policies implementiert werden. Nachfolgend werden die Ergebnisse der einzelnen Zielsetzungen genauer erläutert.

\subsubsection{Definition von Benutzerprofilen}
Die Definition von Benutzerprofilen wurde umgesetzt, hat aber einiges mehr Aufwand gekostet, als geplant war. Mit dem ISE für die Verwaltung der Benutzeridentitäten und Profile, war einiges notwendig, um den ganzen Ablauf des Policy Enforcements zu verstehen.

\subsubsection{Benutzermobilität}
Nach dem erfolgreichen umsetzten der Definition von Benutzerprofilen, konnte in einem weiteren Schritt auch die Benutzermobilität umgesetzt werden. Diese Benutzermobilität funktionierte bei einem Test reibungslos und unglaublich schnell. Nach dem umstecken erfolgte lediglich ein Packet Loss und es blieb sogar eine bestehende SSH Verbindung erhalten. 

%eventuell Screenshot von LISP bei der Benutzermobilität um zu zeigen das nur ein Packet Loss und Verbindung schnell wieder vorhanden.

\subsubsection{Reporting der Netzwerkaktivitäten}
Mit Hilfe der DNA Center API können regelmässige Reports über den Zustand der Netzwerkumgebung per E-Mail versendet werden. Es konnte in der Arbeit ein sehr rudimentäres Reporting implementiert werden, mit dem lediglich eine Liste aller Netzwerkgeräte, sowie eine Liste aller Hosts mit den wichtigsten Informationen, per E-Mail versendet wird. Leider unterstützt die API des aktuellen Release 1.1.6 noch keine Reportingfunktionen im Assurance Bereich. Diese Erweiterung ist erst im Release 1.2 implementiert und wird aber nach wie vor als EFT gekennzeichnet.

\subsubsection{Degradation der Infrastruktur}
Die Degradation konnte aus Zeitgründen leider nicht mehr umgesetzt und getestet werden. Es sind aber Informationen von Cisco Experten vorhanden, welche das DNA Center ausgiebig getestet und teilweise schon bei einigen Kunden implementiert haben, mit welchen ein guter Einblick gewonnen werden kann. Die Switches, sowie Server sollten in einem Netzwerk auf jeden Fall Redundant vorhanden sein, damit ein Ausfall keine weiterreichende Probleme verursacht. Der Ausfall des DNA Centers sollte keinen direkten Einfluss auf das Netzwerk haben. Es können allerdings keine zentralen Konfiguration mehr gemacht werden und die Assurance Daten stehen nicht mehr zur Verfügung. Fällt ein ISE aus und es ist kein weiterer ISE mehr vorhanden, können sich die Benutzer nicht mehr am Netzwerk authentifizieren und die Policies sind nicht mehr verfügbar, da diese allesamt nicht im DNA Center, sondern auf dem ISE erfasst werden.

\subsubsection{Backup und Restore}
Das Backup und Restore funktionierte. Jedoch haben wir beim lesen der Release Notes 1.2 des DNA Center gemerkt, das vorher anscheinend der Assurance Teil noch gar nicht im Backup inkludiert war. In einer früheren Version des DNA Centers funktionierte das Backup aus unerfindlichen Gründen nicht und brachte das gesamte DNA Center zum abstürzen. Dies zeigte sich, in dem nach und nach immer mehr Docker Container abgestürzt und aus diesem Grund auch Teile des DNA Centers nur noch teilweise bis gar nicht mehr reagiert haben. Das abstürzen der Docker Container ging so weiter, bis das DNA Center gar nicht mehr erreichbar und in einem Kong Error stagnierte. In dem Release 1.1.7 funktionierte das Backup nach mehrmaligem Hinzufügen eines Backup Servers jedoch ohne weiteres und konnte auch wiederhergestellt werden. Es ist zum jetzigen Zeitpunkt aber nicht möglich eine Auswahl über die zu sichernden Elemente zu treffen.

\subsubsection{Anbindung an externe Systeme wie die Identity Services Engine (ISE) und Infoblox}
Die externen Systeme, welche in dieser Arbeit verwendet wurden, konnten grösstenweise gut an das DNA Center angebunden werden. Dies war aber auch nur möglich, wenn die genauen empfohlenen Versionen von Cisco eingehalten wurden. So musste beispielsweise der ISE bei einer Neuinstallation eine Version heruntergestuft werden, um die volle Funktionalität und Synchronisation mit dem DNA Center sicherzustellen. Die Anbindung des Infoblox hat ohne Probleme funktioniert, jedoch funktioniert die Kommunikation einzelner Elemente, wie zum Beispiel der IP Adress Pools nur vom DNA Center zum Infoblox. Umgekehrt funktioniert die Synchronisation der IP Adress Pools gar nicht. 

\subsection{Bugs}
Wie bei neuerer Software üblich, sind auch in DNA Center noch verhältnismässig viele Bugs vorhanden. Die Bugs wurden in dieser Arbeit dokumentiert und jeweils an die Cisco Experten weitergeleitet. Teilweise konnten die Bugs mit neuen Releases behoben werden. Bei einzelnen Bugs sind Antworten oder Verbesserungen noch ausstehend. 



