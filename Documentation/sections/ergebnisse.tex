\section{Ergebnisdiskussion}
Stärken und Schwächen der Konzepte, Verbesserungen für die Zielgruppe im Kontext





\subsection{Definition von Benutzerprofilen}
Die Definition von Benutzerprofilen wurde umgesetzt, hat aber einiges mehr Aufwand gekostet, als geplant war. Mit dem ISE für die Verwaltung der Benutzeridentitäten und Profile, war einiges notwendig, um den ganzen Ablauf des Policy Enforcements zu verstehen.



\subsection{Reporting der Netzwerkaktivitäten}


\subsection{Benutzermobilität}


\subsection{Degradation der Infrastruktur}


\subsection{Backup und Restore}


\subsection{Anbindung an externe Systeme wie die Identity Services Engine (ISE) und Infoblox}
Die externen Systeme, welche in dieser Arbeit verwendet wurden, konnten grösstenweise gut an das DNA Center angebunden werden. Dies war aber auch nur möglich, wenn die genauen empfohlenen Versionen von Cisco eingehalten wurden. So musste beispielsweise der ISE bei einer Neuinstallation eine Version heruntergestuft werden, um die volle Funktionalität und Synchronisation mit dem DNA Center sicherzustellen. Die Anbindung des Infoblox hat ohne Probleme funktioniert, jedoch funktioniert die Kommunikation einzelner Elemente, wie zum Beispiel der IP Adress Pools nur vom DNA Center zum Infoblox. Umgekehrt funktioniert die Synchronisation der IP Adress Pools gar nicht. 





Zur Zeit läuft nur die Rapperswil Seite. Jona wurde noch nicht implementiert, um zuerst auf der Seite von Rapperswil eine laufende Fabric mit Policies zu erstellen.

Eventuell af Traces verweisen was wie funktioniert hat.

Bugs und eventuell noch ausstehende Antworten auf Fragen erwähnen

Verbesserungen in Bezug auf vorhandene Bugs und unsere Grafik in Vorgehen mit Schwierigkeiten

