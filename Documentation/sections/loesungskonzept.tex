\section{Lösungskonzept}

\begin{itemize}	
	\item Dokumentation Architektur und Design (i.d.R. plattformneutral bzw. technologieübergreifend, z.B. in Form von UML-Diagrammen und Erläuterungen dazu) 
	\item Architekturentscheidungen mit Begründungen
	\item Diskussion, wie Qualitätsattribute adressiert werden (welche Qualität kann erreicht werden?)
\end{itemize}

\subsection{Technologien}
\subsubsection{Software-Defined Network (SDN)}
SDN ist ein neues Konzept, um Netzwerke zu designen, implementieren und betreiben. Zu den Grundprinzipien des SDN gehören die Entkopplung von Control Plane und Data Plane. Dabei wird die Kontrolle von der Hardware entkoppelt und an eine Software-Anwendung (den Controller) übergeben. Darüber hinaus werden die Netzwerkinfrastruktur und Netzwerkanwendungen getrennt. SDN gibt den Netzadministratoren eine programmierbare, zentrale Steuerung des Netzverkehrs, ohne manuell Zugriff auf die einzelnen physischen Netzkomponenten nehmen zu müssen. Zur Implementierung eines SDN gibt es drei Ansätze:
\begin{itemize}
	\item Switch-basiertes Modell
	\item Overlay-Modell
	\item Hybrid-Ansatz
\end{itemize}

\subsubsection{Cisco Digital Network Architecture Center (Cisco DNA-Center)}

\subsubsection{Locator ID Separation Protocol (LISP)}
LISP ist das Produkt einer Arbeitsgruppe in der Internet Engineering Taskforce (IETF), um was wachsende Problem des doppelten Verwendungszwecks der IP-Adressen zu bereinigen. Zur Zeit wird die IP-Adresse benutzt um die Identität eines Hosts festzulegen und auch den Ort zu bestimmen, an dem er sich im Internet befindet. Dies hat zur Folge das sich bei einem Aufenthalsortwechsel auch die IP-Adresse des Hosts ändert, was bedeutet das die Identität verloren geht und die alten IP-Verbindungen verfallen. \\

Dies soll nun durch LISP geändert werden, in dem es die Identität eines Gerätes, auch Endpoint Identifier (EID) genannt, von seinem Aufenthaltsort, auch Routing Locator (RLOC) genannt, in zwei separate Adressräume unterteilt. Das bedeutet, dass die Router in einer LISP-Architektur nur Routing-Informationen von RLOCs speichern müssen. Um Pfadinformationen eines Hosts abzurufen, kann der Router diese beim LISP-Mapping-Server abfragen, was analog wie das DNS-Mapping funktioniert. \\

LISP verwendet für SDA/Fabric eine VXLAN-Kapselung.

\begin{table}[H]
	\rowcolors{2}{gray!25}{white}
	\centering
	\begin{tabularx}{\textwidth}{p{6.6cm} | X}
		\rowcolor{gray!50}
		\textbf{LISP Device} & \textbf{Function} \\
		\hline	
		ALT (Alternative Logical Topology) & Collects EID data from Map Servers (MS) and advertise aggregate EID prefix. In a deployment of multiple Map Servers, it keeps all synchronized. \\
		
		ETR (Egress Tunnel Router) and PETR (Proxy ETR) & Connects a LISP capable core network. Registers EID prefices with Map Server (MS). Decapsulates LISP packets, received from LISP core. Responds to Map-request messages with a Map-Reply by giving appropriate EID prefix. Typically, this is a CPE (customer premise equipment) router. PETR works on behalf on non-LISP domain and provides LISP-non-LISP connectivity. \\ 
		
		ITR (Ingress Tunnel Router) and PITR (Proxy Ingress Tunnel Router) & Responsible for forwarding local traffic to external destinations. Resolves RLOC for a given destination by sending Map-request to Map Resolver. Encapsulates (vxlan) traffic with LISP header. Typically, this is a Access Layer Switch. PITR works on behalf on non-LISP domain and provides LISP-non-LISP connectivity. \\
		
		XTR (X Tunnel Router) & When both ITR and ETR functions are handled by one router, it is called XTR. This is typical in practice. \\
		
		MR (Map Resolver) & Responds to Map-requests from ITR. Map-requests will be replied with a Negative Map-Reply or forwarded to appropriate ETR or ALT. \\
		
		MS (Map Server) & Registers EID space upon receiving Map-register messages from ETR. Updates ALT and MR with EID and RLOC data. \\
		
		MSMR (Map Server Map Reloader) & When a device acts as both Map Server and Map Resolver, it is called MSMR. This is typical in practice. \\
		
		EID (Endpoint ID) & Endpoint Identifier. IP addresses hidden from core network routing table. RLOC acts next-hop to reach EID space. \\
		
		RLOC (Routing Locator) & Routing Locator. Exists in global routing tables. Authoritative to reach EID space. \\
		               
	\end{tabularx}
	\caption{LISP Elements}
	\label{tab:my-label}
\end{table}

\subsubsection{Virtual Extensible LAN (VXLAN)}
VXLAN ist ein Encapsulation-Protokoll, um ein Overlay-Netzwerk auf einer existierenden Layer 3 Infrasturktur laufen zu lassen. VXLAN wurde ursprünglich von Cisco Systems, VMware und Arista Network entwickelt und ist einer der IETF festgelegten Standards (RFC 7348). \\
\\
Technisch gesehen erzeugt ein VXLAN logische Layer 2 Netzwerke, die dann in standardmässige Layer 3 Pakete eingepackt werden. VXLAN dient dazu um in sehr grossen Netzwerkumgebung die Probleme zu lösen, die durch beschränkte Anzahl von VLANS betroffen sind. Mit VXLAN sind insgesamt 16’777’215 (24 Bit) Layer-2-Umgebungen möglich, die ihrerseits wieder jeweils 4096 VLANs beinhalten können.