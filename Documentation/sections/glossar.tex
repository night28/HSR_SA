\section{Abkürzungsverzeichnis}
\begin{acronym}[SEPSEPSEP]
	\acro{AAA}{Authentication, Authorization, and Accounting}
	\acro{ACI}{Application Centric Infrastructure}
	\acro{ACL}{Access Control List}
	\acro{AP}{Access Point}
	\acro{API}{Application Programming Interface}
	\acro{APIC}{Application Policy Infrastructure Controller}
	\acro{BGP}{Border Gateway Protocol}
	\acro{CAPWAP}{Control and Provisioning of Wireless Access Points}
	\acro{CCO}{Cisco Connection On-line}
	\acro{CLI}{Command-Line Interface}
	\acro{CMD}{Cisco Meta Data}
	\acro{DHCP}{Dynamic Host Configuration Protocol}
	\acro{DMVPN}{Dynamic Multipoint VPN}
	\acro{DNA}{Cisco Digital Network Architecture}
	\acro{DNS}{Domain Name System}
	\acro{EID}{Endpoint Identifier}
	\acro{GRE}{Generic Routing Encapsulation}
	\acro{GUI}{Graphical User Interface}
	\acro{GW}{Gateway}
	\acro{HTDB}{Host Tracking Database}
	\acro{IETF}{Internet Engineering Taskforce}
	\acro{IGP}{Interior Gateway Protocol}
	\acro{IOS}{Internetworking Operating System}
	\acro{IP}{Internet Protocol}
	\acro{IPAM}{IP-Adress-Management}
	\acro{ISE}{Cisco Identity Services Engine}
	\acro{IS-IS}{Intermediate System to Intermediate System}
	\acro{L2}{Layer 2}
	\acro{L3}{Layer 3}
	\acro{LAN}{Local Area Network}
	\acro{LISP}{Locator/ID Separation Protocol}
	\acro{MPLS}{Multiprotocol Label Switching}
	\acro{MR}{Map Resolver}
	\acro{MS}{Map Server}
	\acro{MTU}{Maximum Transmission Unit }
	\acro{NDP}{Network Data Plattfor}
	\acro{PnP}{Plug and Play}
	\acro{pxGrid}{Platform Exchange Grid}
	\acro{RADIUS}{Remote Authentication Dial-In User Service}
	\acro{REST}{Representational State Transfer}
	\acro{RLOC}{Routing locator}
	\acro{SDA}{Software-Defined Access}
	\acro{SDN}{Software-Defined Networking}
	\acro{SGACL}{Scalable Group Access Control List}
	\acro{SGT}{Security Group Tags}
	\acro{SNMP}{Simple Network Management Protocol}
	\acro{SXP}{Security Group Tag Exchange Protocol}
	\acro{VLAN}{Virtual Local Area Network}
	\acro{VN}{Virtual Network}
	\acro{VNI}{Virtual Extensible LAN Network Identifier}
	\acro{VPN}{Virtual Private Network}
	\acro{VRF}{Virtual Routing and Forwarding}
	\acro{VTEP}{Virtual Extensible LAN Tunnel Endpoint}
	\acro{VXLAN}{Virtual Extensible LAN}
	\acro{WAN}{Wide Area Network}
	\acro{WLAN}{Wireless Local Area Network}
\end{acronym}

%\ac{Kuerzel} Bei der ersten Verwendung von \ac{Kuerzel} wird die Langfassung der Abkürzung und die Abkürzung selbst in Klammern dargestellt. Wird der Befehl \ac{Kuerzel} das nächste mal aufgerufen erschneit nur nocht die Abkürzung.

%\acf{Kuerzel} Mit \acf{Kuerzel} gibt es ein zweites Erstes Mal für diese Abkürzung. Das heißt, sie wird wieder in der Langform und der geklammerten Abkürzung gezeigt.

%\acs{Kuerzel} \acs{Kuerzel} gibt nur die Abkürzung aus.

%\acl{Kuerzel} \acl{Kuerzel} gibt nur die Langform der Abkürzung aus.





