\section{Abkürzungsverzeichnis}
\begin{acronym}[SEPSEPSEP]
	\acro{AAA}{authentication, authorization, and accounting}
	\acro{ACL}{access control list}
	\acro{AP}{access point}
	\acro{BGP}{border gateway protocol}
	\acro{CAPWAP}{control and provisioning of wireless access points protocol}
	\acro{CMD}{Cisco Meta Data}
	\acro{DNA}{Cisco Digital Network Architecture}
	\acro{EID}{endpoint identifier}
	\acro{HTDB}{host tracking database}
	\acro{IGP}{interior gateway protocol}
	\acro{IPAM}{IP-Adress-Management}
	\acro{ISE}{Cisco Identity Services Engine}
	\acro{LAN}{Local Area Network}
	\acro{LISP}{Locator/ID Separation Protocol}
	\acro{MR}{Map Resolver}
	\acro{MS}{Map Server}
	\acro{MTU}{maximum transmission unit }
	\acro{PnP}{Plug and Play}
	\acro{RLOC}{routing locator}
	\acro{SDA}{Software-Defined Access}
	\acro{SGACL}{scalable group access control list}
	\acro{SGT}{scalable group tag}
	\acro{SXP}{scalable group tag exchange protocol}
	\acro{VLAN}{virtual local area network}
	\acro{VN}{virtual network}
	\acro{VNI}{virtual extensible LAN network identifier}
	\acro{VRF}{virtual routing and forwarding}
	\acro{VTEP}{virtual extensible LAN tunnel endpoin}
	\acro{VXLAN}{virtual extensible LAN}
	\acro{IPAM}{IP Adress Manager}	
\end{acronym}

%\ac{Kuerzel} Bei der ersten Verwendung von \ac{Kuerzel} wird die Langfassung der Abkürzung und die Abkürzung selbst in Klammern dargestellt. Wird der Befehl \ac{Kuerzel} das nächste mal aufgerufen erschneit nur nocht die Abkürzung.

%\acf{Kuerzel} Mit \acf{Kuerzel} gibt es ein zweites Erstes Mal für diese Abkürzung. Das heißt, sie wird wieder in der Langform und der geklammerten Abkürzung gezeigt.

%\acs{Kuerzel} \acs{Kuerzel} gibt nur die Abkürzung aus.

%\acl{Kuerzel} \acl{Kuerzel} gibt nur die Langform der Abkürzung aus.





