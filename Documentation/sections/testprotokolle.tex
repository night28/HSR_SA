\begin{landscape}
\section {Testprotokolle}


\testpttb{UC01: Definierung von Benutzer und Geräteprofilen}{}{
\testptrow{1}{TheBeschreibung}{TheShould}{TheIs}{TheStatus}
}
\pagebreak

\subsection{UC02: Gastzugang}
Gemäss \textit{Sitzungsprotokoll der Sitzung mit Cisco vom 23.05.2018} im Anhang ist "(...)Gast-Netz(...)Nur möglich mit separatem Border exklusiv für Gast Netz $\rightarrow$ Internet(...)". Der Use Case konnte nicht getestet werden, da kein uns kein zusätzlicher Border zu Verfügung stand und andere Use Cases eine höhere Wichtigkeit hatten. 

\testpttb{UC03-1 Backup DNA Center}{Sämtliche Konfigurationen des DNA Centers sollen gebackuped werden, sodass diese im Notfall wiederhergestellt werden können.}{
	\testptrow{1}{Login auf DNA Center}{DNA Center Dashboard wird angezeigt}{DNA Center Dashboard erscheint}{OK}
	\testptrow{2}{Zu den Backup Einstellungen navigieren \textit{Settings $\rightarrow$ System Settings $\rightarrow$ Backup and Restore} }{Backup Einstellungen anzeigen}{Backup Einstellungen erscheinen}{OK}
	\testptrow{3a}{Backup Server hinzufügen via \textit{Add} $\rightarrow$ SSH IP Address: 217.26.58.9, SSH Port: 22, Server Path: /home/dnacenter/backup, Username: dnacenter, Password: xxx, Encryption Passphrase: xxx. Mittels Apply die Eingaben bestätigen.}{Eingaben werden angenommen.}{Eingaben werden meist nicht angenommen und führen zu Absturz des DNA Centers}{NOT OK}
	\testptrow{3b}{Backup Server hinzufügen via \textit{Add} $\rightarrow$ SSH IP Address: 217.26.58.9, SSH Port: 22, Server Path: /home/dnacenter/backup, Username: dnacenter, Password: xxx, Encryption Passphrase: xxx. Mittels Apply die Eingaben bestätigen.}{Eingaben werden angenommen.}{Eingaben werden angenommen.}{OK}
	\testptrow{4b}{Regelmässiges Backup einrichten via \textit{Schedule $\rightarrow$ Add} Schedule Later, Weekday: Wednesday, Time: 10:30 AM. Mittels Schedule die Eingaben bestätigen.}{Eingaben werden angenommen.}{Eingaben werden angenommen.}{OK}
	\testptrow{5b}{Backup wird regelmässig zum definierten Zeitpunkt ausgeführt.}{Backup wird zum definierten Zeitpunkt ausgeführt}{Backup wird nicht ausgeführt}{NOT OK}
}

\testpttb{UC03-2 Restore DNA Center}{Sämtliche Konfigurationen des DNA Centers sollen aus einem zuvor erstellten Backup wiederhergestellt werden.}{
	\testptrow{1}{Login auf DNA Center}{DNA Center Dashboard wird angezeigt}{DNA Center Dashboard erscheint}{OK}
	\testptrow{2}{Zu den Backup Einstellungen navigieren \textit{Settings $\rightarrow$ System Settings $\rightarrow$ Backup and Restore} }{Zuvor erstellte Backups werden angezeigt}{Backups werden angezeigt}{OK}
	\testptrow{3}{Restore erstellen via \textit{Restore} neben dem gewünschten Backup}{DNA Center wird auf den Stand vom gewählten Backup zurückgesetzt}{DNA Center wurde auf den gewünschten Stand zurückgesetzt}{OK}
}

\subsubsection{Zusammenfassung UC03}
Es kann ein Backup erstellt werden und auch ein Restore eines zuvor erstellten Backups ist möglich. Leider ist das Erfassen, Bearbeiten und Löschen eines Backupservers enorm unzuverlässig und hat mehrfach zu kompletten Abstürzen des DNA Centers geführt. \\
Zudem funktioniert der Backup Schedule nicht. Backups werden nicht automatisch ausgeführt, sind also nur manuell möglich. Auch ein Restore einzelner Komponenten des DNA Centers ist nicht vorgesehen und es gibt kein komplettes Backup des DNA Centers. Einzelne Teile wie z.Bsp. Assurance werden nicht gebackuped. \\
Da die Backup Funktionalität des DNA Centers sehr eingeschränkt ist und nur unzuverlässig funktioniert, wird der Use Case "Backup und Restore" nicht vollständig erfüllt.
\pagebreak

\testpttb{UC04 Reporting}{Mit Hilfe der DNA Center API können regelmässige Reports über den Zustand der Netzwerkumgebung per E-Mail oder Slack versendet werden. Damit dieser Use Case ausgeführt werden kann, muss ein Mailserver und ein Benutzer zur Verfügung stehen, der E-Mails versenden kann. Des weiteren ist ein System benötigt, welches das Script ausführt. Auf diesem muss python installiert sein.}{
	\testptrow{1}{Reporting Script aus GIT Repository auschecken (auf dem System, das die Reports versenden soll)}{Code ist ausgecheckt}{Code ist ausgecheckt}{OK}
	\testptrow{2}{config.py mit Texteditor öffnen und anpassen}{Reporting Config ist komplett}{Reporting Config ist komplett}{OK}
	\testptrow{3}{Cronjob einrichten, der das Script in regelmässigen Abständen ausführt}{Script wird regelmässig ausgeführt}{Script wird regelmässig ausgeführt}{OK}
	\testptrow{4a}{Cronjob wird ausgeführt und versendet Report per E-Mail}{Report wird per E-Mail versendet.}{Report wird per E-Mail versendet}{OK}
	\testptrow{4b}{Cronjob wird ausgeführt und versendet Report per Slack}{Report wird per Slack versendet.}{Nicht implementiert}{NOT OK}
}

\subsubsection{Zusammenfassung UC04}
Mit dieser Lösung ist ein sehr rudimentäres Reporting implementiert worden. Es wird lediglich eine Liste aller Netzwerkgeräte, sowie eine Liste aller Hosts mit den wichtigsten Informationen und dem Zustand der Geräte ausgegeben. Wünschenswert wären natürlich wesentlich mehr Informationen, insbesondere aus dem Bereich Assurance. Leider unterstützt die API des aktuellen Release 1.1.6 diese Funktionien nicht. Im Release 1.2 ist einiges mehr vorhanden, aber nach wie vor als Early Field Trial (EFT) gekennzeichnet. 
Eine sinnvolle Reporting Funktion ist daher mit den aktuell verfügbaren APIs des DNA Centers nicht realisierbar.


\subsection{UC05: Hardware Ersatz}
\textit{Offene Frage an Ivan Caduff via Slack}



\subsection{UC06: Benutzermobilität}
Um diesen Use Case zu testen braucht es eine funktionierende Fabric, zwei Gebäude oder besser zwei Standorte und 802.1X Authentifizierung. Da dies leider nicht der Fall ist, konnte dieser Use Case nicht getestet werden. 



\subsection{UC07: Degradation}
In der Main Success Story dieses Use Cases steht "1. Netzwerk funktioniert einwandfrei.". Da dies zu diesem Zeitpunkt noch nicht der Fall ist, konnte dieser Use Case nicht getestet werden. 



\subsection{UC08: Integration von nicht Fabric Komponenten}
Andere Use Cases haben eine höhere Priorisierung erhalten. Aus Zeitmangel konnte dieser Use Case nicht behandelt werden. Insbesondere, weil nebst einer funktionierenden Fabric auch ein Mapping zwischen den SGT im Legacy Netzwerk konfiguriert hätte werden müssen. 


\pagebreak

\testpttb{UC10: Einsatz von SGT}{Erstellen einer neuen Scalable Group.}{
	\testptrow{1}{Login auf DNA Center}{DNA Center Dashboard wird angezeigt}{DNA Center Dashbaord erscheint}{OK}
	\testptrow{2}{Zu den Einstellungen navigieren \textit{Policy  $\rightarrow$ Registry $\rightarrow$ Scalable Groups }}{Liste der \textit{Scalable Groups} wird angezeigt, inklusive der \textit{Add Group} Schaltfläche.}{Liste der \textit{Scalable Groups} wird angezeigt, inklusive der \textit{Add Group} Schaltfläche.}{OK}
	\testptrow{2}{Hinzufügen einer neuer Gruppe mithilfe der Schaltfläche \textit{Add Group}}{Neuer Dialog erscheint zum Anlegen einer \textit{Scalable Group.}}{Weiterleitung zum Cisco ISE zur Ansicht \textit{Components $\rightarrow$ Security Groups.} (Eventuell muss man sich zuvor beim ISE zusätzlich einloggen.) Dort muss \textit{Add} ausgewählt werden. Es erscheint ein Dialog zum Hinzufügen einer \textit{Security Group.}}{NOT OK}
	\testptrow{3}{Neue Gruppe anlegen mit einem Namen}{Im Dialog kann ein neuer Name für die \textit{Scalable Group} eingegeben werden. Der Dialog wird mit \textit{Speichern} geschlossen. }{Ein neuer Name, ein Symbol und eine Beschreibung kann hinterlegt werden. Zusätzlich muss \textit{Propagate to ACI} angewählt werden. Der Dialog wird mit \textit{Anlegen} geschlossen}{OK}
	\testptrow{4}{Die Scalable Group ist erstellt.}{In der Liste der Scalable Groups wird die neu erstellte Gruppe angezeigt.}{Da man immernoch in der ISE Ansicht ist, muss zuerst zum alten Tab des DNA Centers gewechselt werden. Anschliessend muss die Seite neu geladen werden. Die neue erstellte Gruppe wird angezeigt.}{OK}
}
\subsubsection{Zusammenfassung UC10}
Noch nicht alle Funktionen können komplett im DNA Center erledigt werden. Ein Teil der Funktionen erfolgt weiterhin über die GUI der anderen Komponenten. 

\pagebreak

\testpttb{UC11-1: Infoblox verknüpfen}{Durch die Integration des Infoblox DDI im DNA Center soll das IP-Adressen Management für neue Netzwerkkomponenten vereinfacht werden.}{
	\testptrow{1}{Login auf DNA Center}{DNA Center Dashboard wird angezeigt}{DNA Center Dashboard erscheint}{OK}
	\testptrow{2}{Zu IP Adress Manager Einstellungen navigieren über \textit{Settings $\rightarrow$ System Settings $\rightarrow$ Settings $\rightarrow$ IP Adress Manager} }{IP Adress Manager Einstellungen anzeigen}{IP Adress Manager Einstellungen erscheinen}{OK}
	\testptrow{3}{Infoblox Informationen hinterlegen $\rightarrow$ Server Name: Infoblox, Server Url: https://10.22.0.21, Username: admin, Password: xxx, Provider: INFOBLOX). Mittels Apply die Eingaben bestätigen.}{Eingaben werden angenommen.}{Eingaben wurden angenommen und Verbindung zu Infoblox Server erfolgreich hergestellt.}{OK}
}

\testpttb{UC11-2: IP Adress Pool erstellen}{IP Adress Pools auf dem Infoblox erstellen und mit DNA Center synchronisieren}{
	\testptrow{1}{IP Adress Pools anzeigen über \textit{Design $\rightarrow$ Network Settings $\rightarrow$ IP Adress Pools}.}{IP Adress Pools sollten angezeigt werden.}{Es werden vorhandene IP Adress Pools angezeigt.}{OK}
	\testptrow{2}{Mit einem Klick auf \textit{Add IP Pool} kann ein neuer IP Pool hinzugefügt werden. Hierfür werden folgende Angaben benötigt: IP Pool Name, CIDR Prefix, IP Subnetz, Gateway IP Adresse, DHCP Server (optional), DNS Server (optional)}{Fenster um IP Adress Pool hinzuzufügen erscheint}{Fenster um IP Adress Pool hinzuzufügen ist erschienen}{OK}
	\testptrow{3}{Mit einem Klick auf Save wird der IP Adress Pool hinzugefügt und mit dem Infoblox Server synchronisiert}{Übersicht über vorhandene IP Adress Pools wird angezeigt}{Übersicht über vorhandene IP Adress Pools wird angezeigt mit vorher hinzugefügtem IP Adress Pool}{OK}
}

\subsubsection{Zusammenfassung UC11}
Der Infoblox konnte im DNA Center relativ einfach hinterlegt werden. Das erstellen eines IP Adress Pools auf dem DNA Center funktioniert gut und wird auch schnell mit dem Infoblox synchronisiert. Wird jedoch ein IP Adress Pool auf dem Infoblox erstellt, so ist es eher mühsam diesen neuen IP Adress Pool auf dem DNA Center anzuzeigen. Diese Synchronisation erfolgt nicht sauber. Es ist auf dem DNA Center zwar eine Import Funktion vorhanden, welche die IP Adress Pools vom Infoblox importieren sollte, aber nur mässig gut funktioniert. Jedes Netz welches importiert werden soll, muss einzeln erstellt und genaustens angegeben werden. Aus diesem Grunde sollte generell alles was im DNA Center erstellt werden kann, auf diesem erstellt werden und nicht manuell auf dem Infoblox.

\pagebreak


\end{landscape}