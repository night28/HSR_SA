\begin{landscape}
\section {Testprotokolle}

\testpttb{UC01-1: Anlegen von Benutzern}{}{
\testptrow{1}{Login in ISE}{Eingelogged in ISE}{Eingelogged in ISE}{OK}
\testptrow{2}{Benutzer anlegen via \textit{Work Centers $\rightarrow$ Network Access $\rightarrow$ Identities $\rightarrow$ Network Access Users}}{User ist erstellt}{User ist erstellt}{OK}
\testptrow{2}{Benutzer einer Gruppe zuweisen via \textit{Work Centers $\rightarrow$ Network Access $\rightarrow$ Identities $\rightarrow$ Network Access Users}}{User ist in Gruppe}{User ist in Gruppe}{OK}
}

Die Benutzer sind nun angelegt, ein Login ist aber noch nicht möglich, da noch keine Authentication Policy für diese erstellt wurde. Dies wird daher im nächsten Schritt erledigt.

\testpttb{UC01-2: Anlegen von Authentication Policies}{}{
	\testptrow{1}{Login in ISE}{Eingelogged in ISE}{Eingelogged in ISE}{OK}
	\testptrow{2}{Authentication Policy anlegen via \textit{Work Centers $\rightarrow$ Network Access $\rightarrow$ Policy Sets $\rightarrow$ Default} für 802.1x Wired}{Policy ist erstellt}{Policy ist erstellt}{OK}
	\testptrow{3}{User logged sich ein}{User ist eingelogged}{User ist eingelogged}{OK}
}

Nun kann sich der Benutzer einloggen, wird aber noch nicht einer Scalable Group zugewiesen. Somit hat er keine für ihn gültige Policy und somit keine Zugriffe im Netzwerk. Damit der Benutzer die korrekten Policies erhält, muss eine Authorization Policy erstellt werden.

\testpttb{UC01-3: Anlegen von Authorization Policies}{}{
	\testptrow{1}{Login in ISE}{Eingelogged in ISE}{Eingelogged in ISE}{OK}
	\testptrow{2}{Authorization Policy anlegen via \textit{Work Centers $\rightarrow$ Network Access $\rightarrow$ Policy Sets $\rightarrow$ Default} für 802.1x Wired und Gruppe des Users}{Authorization Policy ist erstellt}{Authorization Policy ist erstellt}{OK}
	\testptrow{3}{User wird beim Login der korrekten SG zugewiesen}{User ist der korrekten Gruppe zugewiesen}{User ist der korrekten Gruppe zugewiesen}{OK}
}

\testpttb{UC01-4: Einrichten einer Policy}{}{
	\testptrow{1}{Login auf DNA Center}{DNA Center Dashboard wird angezeigt}{DNA Center Dashboard wird angezeigt}{OK}
	\testptrow{2}{Policy einrichten unter \textit{Policy $\rightarrow$ Policy Administration $\rightarrow$ Group Based Access Control}}{Policy wird angelegt}{Policy wird angelegt}{OK}
	\testptrow{3}{DNA Center synchronisiert Policy mit dem ISE}{Policy ist auf ISE}{Policy ist auf ISE}{OK}
	\testptrow{4}{ISE pushed Policy auf Border und Edge Nodes}{Policy ist auf Nodes vorhanden}{Policy ist auf Nodes vorhanden}{OK}
	\testptrow{5}{Die Nodes enforcen die Policy für die korrekten Clients}{Policy wird enforced}{Policy wird enforced}{OK}
}

Die Policy ist nun auf allen nötigen Geräten deployed und wird von diesen enforced. Die Clients dürfen also nur auf die Ressourcen zugreifen, die gemäss der definierten Policy erlaubt sind.
\pagebreak

\testpttb{UC02-1 Backup DNA Center}{Sämtliche Konfigurationen des DNA Centers sollen gebackuped werden, sodass diese im Notfall wiederhergestellt werden können.}{
	\testptrow{1}{Login auf DNA Center}{DNA Center Dashboard wird angezeigt}{DNA Center Dashboard erscheint}{OK}
	\testptrow{2}{Zu den Backup Einstellungen navigieren \textit{Settings $\rightarrow$ System Settings $\rightarrow$ Backup and Restore} }{Backup Einstellungen anzeigen}{Backup Einstellungen erscheinen}{OK}
	\testptrow{3a}{Backup Server hinzufügen via \textit{Add} $\rightarrow$ SSH IP Address: 217.26.58.9, SSH Port: 22, Server Path: /home/dnacenter/backup, Username: dnacenter, Password: xxx, Encryption Passphrase: xxx. Mittels Apply die Eingaben bestätigen.}{Eingaben werden angenommen.}{Eingaben werden meist nicht angenommen und führen zu Absturz des DNA Centers}{NOT OK}
	\testptrow{3b}{Backup Server hinzufügen via \textit{Add} $\rightarrow$ SSH IP Address: 217.26.58.9, SSH Port: 22, Server Path: /home/dnacenter/backup, Username: dnacenter, Password: xxx, Encryption Passphrase: xxx. Mittels Apply die Eingaben bestätigen.}{Eingaben werden angenommen.}{Eingaben werden angenommen.}{OK}
	\testptrow{4b}{Regelmässiges Backup einrichten via \textit{Schedule $\rightarrow$ Add} Schedule Later, Weekday: Wednesday, Time: 10:30 AM. Mittels Schedule die Eingaben bestätigen.}{Eingaben werden angenommen.}{Eingaben werden angenommen.}{OK}
	\testptrow{5b}{Backup wird regelmässig zum definierten Zeitpunkt ausgeführt.}{Backup wird zum definierten Zeitpunkt ausgeführt}{Backup wird nicht ausgeführt}{NOT OK}
}

\testpttb{UC02-2 Restore DNA Center}{Sämtliche Konfigurationen des DNA Centers sollen aus einem zuvor erstellten Backup wiederhergestellt werden.}{
	\testptrow{1}{Login auf DNA Center}{DNA Center Dashboard wird angezeigt}{DNA Center Dashboard erscheint}{OK}
	\testptrow{2}{Zu den Backup Einstellungen navigieren \textit{Settings $\rightarrow$ System Settings $\rightarrow$ Backup and Restore} }{Zuvor erstellte Backups werden angezeigt}{Backups werden angezeigt}{OK}
	\testptrow{3}{Restore erstellen via \textit{Restore} neben dem gewünschten Backup}{DNA Center wird auf den Stand vom gewählten Backup zurückgesetzt}{DNA Center wurde auf den gewünschten Stand zurückgesetzt}{OK}
}

\subsubsection{Zusammenfassung UC02}
Es kann ein Backup erstellt werden und auch ein Restore eines zuvor erstellten Backups ist möglich. Leider ist das Erfassen, Bearbeiten und Löschen eines Backupservers enorm unzuverlässig und hat mehrfach zu kompletten Abstürzen des DNA Centers geführt. \\
Zudem funktioniert der Backup Schedule nicht. Backups werden nicht automatisch ausgeführt, sind also nur manuell möglich. Auch ein Restore einzelner Komponenten des DNA Centers ist nicht vorgesehen und es gibt kein komplettes Backup des DNA Centers. Einzelne Teile wie z.Bsp. Assurance werden nicht gebackuped. \\
Da die Backup Funktionalität des DNA Centers sehr eingeschränkt ist und nur unzuverlässig funktioniert, wird der Use Case "Backup und Restore" nicht vollständig erfüllt.
\pagebreak

\testpttb{UC03 Reporting}{Mit Hilfe der DNA Center API können regelmässige Reports über den Zustand der Netzwerkumgebung per E-Mail oder Slack versendet werden. Damit dieser Use Case ausgeführt werden kann, muss ein Mailserver und ein Benutzer zur Verfügung stehen, der E-Mails versenden kann. Des weiteren ist ein System benötigt, welches das Script ausführt. Auf diesem muss python installiert sein.}{
	\testptrow{1}{Reporting Script aus GIT Repository auschecken (auf dem System, das die Reports versenden soll)}{Code ist ausgecheckt}{Code ist ausgecheckt}{OK}
	\testptrow{2}{config.py mit Texteditor öffnen und anpassen}{Reporting Config ist komplett}{Reporting Config ist komplett}{OK}
	\testptrow{3}{Cronjob einrichten, der das Script in regelmässigen Abständen ausführt}{Script wird regelmässig ausgeführt}{Script wird regelmässig ausgeführt}{OK}
	\testptrow{4a}{Cronjob wird ausgeführt und versendet Report per E-Mail}{Report wird per E-Mail versendet.}{Report wird per E-Mail versendet}{OK}
	\testptrow{4b}{Cronjob wird ausgeführt und versendet Report per Slack}{Report wird per Slack versendet.}{Nicht implementiert}{NOT OK}
}

\subsubsection{Zusammenfassung UC03}
Mit dieser Lösung ist ein sehr rudimentäres Reporting implementiert worden. Es wird lediglich eine Liste aller Netzwerkgeräte, sowie eine Liste aller Hosts mit den wichtigsten Informationen und dem Zustand der Geräte ausgegeben. Wünschenswert wären natürlich wesentlich mehr Informationen, insbesondere aus dem Bereich Assurance. Leider unterstützt die API des aktuellen Release 1.1.6 diese Funktionien nicht. Im Release 1.2 ist einiges mehr vorhanden, aber nach wie vor als Early Field Trial (EFT) gekennzeichnet. 
Eine sinnvolle Reporting Funktion ist daher mit den aktuell verfügbaren APIs des DNA Centers nicht realisierbar.
\pagebreak

\subsection{UC04: Hardware Ersatz}
Wenn ein Device in der Fabric ausfällt, ist folgendermassen vorzugehen, um dieses auszutauschen und den alten Stand der Fabric wiederherzustellen.\\
\begin{itemize}
	\item Altes Gerät durch neues ersetzen
	\item Neues Gerät identisch verkabeln
	\item Das Gerät mittels LAN Automation wieder in Betrieb nehmen
	\item Device Provisioning ausführen
	\item Device der ursprünglichen Fabric hinzufügen
	\item Device Provisioning erneut ausführen
	\subitem Sofern der Gerättyp zwischen altem und neuen Device identisch ist, sollte die Access Portkonfiguration übernommen werden
	\subitem Sind es unterschiedliche Gerätetypen müssen die Access Ports manuell konfiguriert werden
	\item Der neue Switch sollte nun die komplette Funktion des alten übernommen haben\\
\end{itemize}

Die obigen Informationen sind nur theoretisch und konnten in der Testumgebung nicht getestet werden.
\pagebreak


\testpttb{UC05 Benutzermobilität}{Ein User ändert seinen Arbeitsplatz, das Gebäude oder den Arbeitsort. Er muss an allen Standorten dieselben Policies erhalten und auf dieselben Ressourcen zugreifen können.}{
	\testptrow{1}{User ist in Gebäude 1 mit dem Netzwerk verbunden}{User ist mit dem Netzwerk verbunden}{User ist mit dem Netzwerk verbunden}{OK}
	\testptrow{2}{User authentifiziert sich und erhält die korrekten Policies}{User erreicht nur die Ressourcen, die per Policy erlaubt sind}{User erreicht nur die Ressourcen, die per Policy erlaubt sind}{OK}
	\testptrow{3}{User trennt Verbindung in Gebäude 1}{Verbindung ist getrennt}{Verbindung ist getrennt}{OK}
	\testptrow{4}{User verbindet sich im Gebäude 2}{User ist mit dem Netzwerk verbunden}{User ist mit dem Netzwerk verbunden}{OK}
	\testptrow{5}{User authentifiziert sich}{User ist authentifiziert}{User ist authentifiziert}{OK}
	\testptrow{6}{User erhält die korrekten Policies}{User erreicht dieselben Ressourcen wie zuvor in Gebäude 1}{User erreicht dieselben Ressourcen wie zuvor n Gebäude 1}{OK}
}

\subsubsection{Zusammenfassung UC05}
Benutzermobilität funktioniert in einer vom DNA Center verwalteten sehr gut. In unseren Tests wurden sogar bestehende Sessions vom alten an den neuen Standort übernommen. 
\pagebreak
\subsection{UC06: Degradation}
In diesem Use Case wird beschrieben, wie sich das Netzwerk verhält, wenn einzelne Komponenten ausfallen. In untenstehender Tabelle ist zu sehen, was für einen Impact der Ausfall der einzelnen Komponenten hat. Die Informationen sind allerdings nur theoretisch und konnten in der Lab Umbegung nicht getestet werden.
\\

\rowcolors{2}{gray!25}{white}
\begin{longtable}{| m{2.5cm} | m{21.5cm} |}
		\hline
		\textbf{Ausfall Gerät}  & \textbf{Impact}        \\
		\hline
		Edge Node  & Alle Clients an diesem Device verlieren die Konnektivität.  \\ 
		\hline
		Intermediate Node & Kein Impact solange alle Edge Devices noch einen Pfad zum Border und somit zu den anderen Edge Nodes haben \\ 
		\hline
		Border Node & Kein Impact, solange ein weiterer Border verhanden ist. Falls nicht, verliert die Fabric die Kommunikation zu Geräten ausserhalb der Fabric \\
		\hline
		Control Plane & Kein Impact, solange eine weitere Control Plane vorhanden ist. Falls nicht, können die Edge Nodes den RLOC von Clients nicht mehr ermitteln und müssen auf den eigenen Cache zurückgreifen. Die Kommunikation innerhalb und ausserhalb der Fabric ist also eingeschränkt. Sobald die TTL der Cache Einträge abgelaufen ist, ist nur noch Kommunikation zwischen Clients am gleichen Edge Node möglich. \\
		\hline
		DNA Center & Ein Ausfall des DNA Centers hat keinen direkten Einfluss auf das Netzwerk. Es können allerdings keine Anpassungen mehr gemacht werden, die Assurance Daten stehen nicht zur Verfügung und werden nicht mehr erfasst. \\
		\hline
		ISE & Fällt der ISE aus und es ist kein weiterer ISE vorhanden, können sich Benutzer nicht mehr am Netzwerk authentifizieren und die Policies sind nicht mehr verfügbar. Das heisst, der Netzwerkbetrieb ist stark eingeschränkt oder gar komplett unterbrochen. Es gibt daher die Möglichkeit, ein "Notfall VLAN" zu definieren, über welches die Clients in diesem Fall weiterhin kommunizieren könnten. \\
		\hline
		Infoblox & Fällt Infoblox aus, können keine neuen IP Pools angelegt werden. Des Weiteren funktioniert DNS nicht mehr, sofern Infoblox der einzige DNS Server war. Dasselbe gilt auch für DHCP. \\
		\hline
		Kabelunter- bruch & Ein Kabelunterbruch zwischen zwei Netzwerkkomponenten in der Fabric hat keinen Impact auf den Betrieb, sofern alle Edge Nodes über Pfade zu den anderen Edge Nodes und dem Border verfügen.\\
		\hline
	\caption{UC06: Testprotokoll Degradation}
	\label{tab:UC06_Testprotokoll}
\end{longtable}
\pagebreak

\subsection{UC07: Integration von nicht Fabric Komponenten}
Der Einsatz von nicht Fabric Komponenten ist im aktuellen Release nicht unterstützt. Es ist aber vorgesehen, dass im Release 1.2 die "SD Access Extension for IoT" eingeführt wird, mit der die Fabric mit Hilfe von Geräten wie zum Beispiel dem C3650CX erweitert werden kann.
\pagebreak

\subsection{UC08: Migration von bestehendem klassischen Campus}
Es gibt derzeit keinen definierten Migrationspfad von einem klassischen Campus Netzwerk. Die Empfehlung von Cisco ist allerdings wie folgt:
\begin{itemize}
	\item Parallel zu bestehendem Netzwerk einen Border Node aufsetzen
	\item Mit Hilfe der LAN Automation einen Intermediate Node in Betrieb nehmen
	\item Mit Hilfe der LAN Automation einen ersten Edge Node in Betrieb nehmen
	\item Fabric erstellen
	\item User, Policies usw. definieren
	\subitem Schrittweise Clients migrieren
	\item Schrittweise weitere Nodes (Border, Intermediate, Edge) hinzufügen
\end{itemize}

Werden Devices aus einem klassischen Netz in die Fabric übernommen, sollten die Konfiguration zuerst gelöscht werden. 
Dieses Vorgehen wird von Cisco so empfohlen, konnte aber in unserer Testumgebung nicht getestet werden, da ein Parallelbetrieb einer klassischen Umgebung und der Fabric zu aufwändig gewesen wäre.
\pagebreak

\testpttb{UC09: Einsatz von SGT}{Erstellen einer neuen Scalable Group.}{
	\testptrow{1}{Login auf DNA Center}{DNA Center Dashboard wird angezeigt}{DNA Center Dashbaord erscheint}{OK}
	\testptrow{2}{Zu den Einstellungen navigieren \textit{Policy  $\rightarrow$ Registry $\rightarrow$ Scalable Groups }}{Liste der \textit{Scalable Groups} wird angezeigt, inklusive der \textit{Add Group} Schaltfläche.}{Liste der \textit{Scalable Groups} wird angezeigt, inklusive der \textit{Add Group} Schaltfläche.}{OK}
	\testptrow{2}{Hinzufügen einer neuer Gruppe mithilfe der Schaltfläche \textit{Add Group}}{Neuer Dialog erscheint zum Anlegen einer \textit{Scalable Group.}}{Weiterleitung zum Cisco ISE zur Ansicht \textit{Components $\rightarrow$ Security Groups.} (Eventuell muss man sich zuvor beim ISE zusätzlich einloggen.) Dort muss \textit{Add} ausgewählt werden. Es erscheint ein Dialog zum Hinzufügen einer \textit{Security Group.}}{NOT OK}
	\testptrow{3}{Neue Gruppe anlegen mit einem Namen}{Im Dialog kann ein neuer Name für die \textit{Scalable Group} eingegeben werden. Der Dialog wird mit \textit{Speichern} geschlossen. }{Ein neuer Name, ein Symbol und eine Beschreibung kann hinterlegt werden. Zusätzlich muss \textit{Propagate to ACI} angewählt werden. Der Dialog wird mit \textit{Anlegen} geschlossen}{OK}
	\testptrow{4}{Die Scalable Group ist erstellt.}{In der Liste der Scalable Groups wird die neu erstellte Gruppe angezeigt.}{Da man immernoch in der ISE Ansicht ist, muss zuerst zum alten Tab des DNA Centers gewechselt werden. Anschliessend muss die Seite neu geladen werden. Die neue erstellte Gruppe wird angezeigt.}{OK}
}

\subsubsection{Zusammenfassung UC09: Einstaz von SGT}
Noch nicht alle Funktionen können komplett im DNA Center erledigt werden. Ein Teil der Funktionen erfolgt weiterhin über die GUI der anderen Komponenten. 

\pagebreak

\testpttb{UC10-1: Infoblox verknüpfen}{Durch die Integration des Infoblox DDI im DNA Center soll das IP-Adressen Management für neue Netzwerkkomponenten vereinfacht werden.}{
	\testptrow{1}{Login auf DNA Center}{DNA Center Dashboard wird angezeigt}{DNA Center Dashboard erscheint}{OK}
	\testptrow{2}{Zu IP Adress Manager Einstellungen navigieren über \textit{Settings $\rightarrow$ System Settings $\rightarrow$ Settings $\rightarrow$ IP Adress Manager} }{IP Adress Manager Einstellungen anzeigen}{IP Adress Manager Einstellungen erscheinen}{OK}
	\testptrow{3}{Infoblox Informationen hinterlegen $\rightarrow$ Server Name: Infoblox, Server Url: https://10.22.0.21, Username: admin, Password: xxx, Provider: INFOBLOX). Mittels Apply die Eingaben bestätigen.}{Eingaben werden angenommen.}{Eingaben wurden angenommen und Verbindung zu Infoblox Server erfolgreich hergestellt.}{OK}
}

\testpttb{UC10-2: IP Adress Pool erstellen}{IP Adress Pools auf dem Infoblox erstellen und mit DNA Center synchronisieren}{
	\testptrow{1}{IP Adress Pools anzeigen über \textit{Design $\rightarrow$ Network Settings $\rightarrow$ IP Adress Pools}.}{IP Adress Pools sollten angezeigt werden.}{Es werden vorhandene IP Adress Pools angezeigt.}{OK}
	\testptrow{2}{Mit einem Klick auf \textit{Add IP Pool} kann ein neuer IP Pool hinzugefügt werden. Hierfür werden folgende Angaben benötigt: IP Pool Name, CIDR Prefix, IP Subnetz, Gateway IP Adresse, DHCP Server (optional), DNS Server (optional)}{Fenster um IP Adress Pool hinzuzufügen erscheint}{Fenster um IP Adress Pool hinzuzufügen ist erschienen}{OK}
	\testptrow{3}{Mit einem Klick auf Save wird der IP Adress Pool hinzugefügt und mit dem Infoblox Server synchronisiert}{Übersicht über vorhandene IP Adress Pools wird angezeigt}{Übersicht über vorhandene IP Adress Pools wird angezeigt mit vorher hinzugefügtem IP Adress Pool}{OK}
}

\subsubsection{Zusammenfassung UC10}
Der Infoblox konnte im DNA Center relativ einfach hinterlegt werden. Das erstellen eines IP Adress Pools auf dem DNA Center funktioniert gut und wird auch schnell mit dem Infoblox synchronisiert. Wird jedoch ein IP Adress Pool auf dem Infoblox erstellt, so ist es eher mühsam diesen neuen IP Adress Pool auf dem DNA Center anzuzeigen. Diese Synchronisation erfolgt nicht sauber. Es ist auf dem DNA Center zwar eine Import Funktion vorhanden, welche die IP Adress Pools vom Infoblox importieren sollte, aber nur mässig gut funktioniert. Jedes Netz welches importiert werden soll, muss einzeln erstellt und genaustens angegeben werden. Aus diesem Grunde sollte generell alles was im DNA Center erstellt werden kann, auf diesem erstellt werden und nicht manuell auf dem Infoblox.

\pagebreak


\end{landscape}